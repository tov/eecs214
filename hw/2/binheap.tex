\documentclass[12pt]{article}

\usepackage{fancyhdr}
\usepackage{fullpage}
\usepackage{hyperref}
\usepackage{microtype}
\usepackage{amsmath,amssymb}
\usepackage[MnSymbol]{mathspec}
\usepackage{mathspec}
\usepackage[final]{listings}

\setlength{\parindent}{0em}
\setlength{\parskip}{0.5em}

\pagestyle{fancy}
\lhead{EECS 214}
\rhead{Fall 2015}
\cfoot{}
\renewcommand{\headrulewidth}{0pt}
\renewcommand{\footrulewidth}{0pt}
\setlength{\headheight}{.25in}
\setlength{\headsep}{.25in}
\addtolength{\topmargin}{-.5in}

\newcommand{\sourcefilelong}{%
  \url{http://users.eecs.northwestern.edu/~jesse/course/eecs214-fa15/hw/2/binheap.rkt}%
}

\newcommand{\sourcefile}{%
  \href{http://users.eecs.northwestern.edu/~jesse/course/eecs214-fa15/hw/2/binheap.rkt}{\texttt{binheap.rkt}}%
}

%%%
%%% LISTINGS SETUP
%%%

\lstdefinelanguage{ASL}%
  {sensitive,%
   alsoletter=-+*/\#?,%
   alsodigit=-,%
   morekeywords={%
      and,%
      begin,%
      check-error,%
      check-expect,%
      check-within,%
      cond,%
      define,%
      define-struct,%
      else,%
      if,%
      lambda,%
      local,%
      or,%
   },%
   morekeywords=[2]{%
     cons,%
     \#true,%
     \#false,%
   },%
   morekeywords=[3]{%
     this-is-here-because-suck,%
     cons?,%
     empty?,%
     -,+,*,/,%
     first,%
     rest,%
   }%,
   morecomment=[l];,%
   morecomment=[s]{\#|}{|\#},%
   morestring=[b]"%
  }[keywords,comments,strings]%

\lstnewenvironment{asl}[1][]
{%
  \begingroup
  \lstset{language=ASL,#1}%
}
{%
  \endgroup
}

\lstnewenvironment{cpp}[1][]
{%
  \begingroup
  \lstset{language=C++,#1}%
}
{%
  \endgroup
}


\newcommand{\textcode}{%
  \lstset{language=ASL,basicstyle=\ttfamily\small}%
  \lstinline
}

\lstset{%
  numberstyle=\scriptsize,
  showspaces=false,
  basicstyle=\ttfamily,
  keywordstyle=\ttfamily\bfseries,
  %keywordstyle=[2]\ttfamily\color{blue!70!black},
  %keywordstyle=[3]\ttfamily\color{green!40!black},
  numberstyle=\ttfamily,%\color{blue!70!black},
  commentstyle=\itshape,%\color{violet!80!black},
  escapeinside={($}{$)},
}

\newcommand{\textasl}{%
  \lstset{language=ASL,basicstyle=\ttfamily\small}%
  \lstinline
}

\let\sup^

\catcode`\^\active\relax
\def^#1^{\textasl|#1|}

\begin{document}
\begin{center}
  {\LARGE\textbf{HW2: Binary Heaps}}
  \par
  {\large \textbf{Due:} Monday, November 16, at 11:59 PM, on Canvas}
\end{center}

\textbf{You may work on your own or with one (1) partner.}
\par\bigskip

In this assignment, you will implement a fixed-size binary heap. The
structure of the heap is already defined for you in
\sourcefile\footnote{\sourcefilelong}. The heap is represented using an
ASL vector\footnote{ASL vectors are like other languages' arrays in that
the size is fixed at create time.} to contain the elements.  Each heap
will also contain a comparison function for ordering the elements of the
heap, so that your implementation can support heaps of integers, heaps
of strings, heaps of whatits, heaps of sporkles, etc.

\subsection*{Your task}

In \sourcefile, I've supplied a definition of a function ^create^ that
returns a new, empty heap given a capacity and ordering function.
Implementing the remaining operations is up to you:

\begin{tabular}{l@{\texttt{ : }}l}
 ^insert!^     & ^[Heap X] X -> Void^ \\
 ^find-min^    & ^[Heap X] -> X^ \\
 ^remove-min!^ & ^[Heap X] -> Void^ \\
\end{tabular}

For details, see the function headers provided in \sourcefile, which
include purpose statements as well. Each operation must have worst-case
time complexity $\mathcal O(\log n)$, where $n$ is the number of
elements in the heap. In order to help you factor your program
effectively, I've included at the bottom of \sourcefile{} a list of
helper functions with names, signatures (types), and purpose statements
(brief functional descriptions). You are free to use as much or as
little of my design as you like.

\subsection*{Extra credit}

Make your heap expand as necessary to accomodate any number of assertions.
To achieve this, instead of failing when the heap is full, ^insert!^ should
allocate a new vector that doubles the capacity and copy the existing element
over from the old vector.

\subsection*{Deliverables}

\begin{itemize}
  \item The provided file \sourcefile, containing:
    \begin{itemize}
      \item the ^insert!^, ^find-min^, and ^remove-min!^ functions fully defined, and
      \item sufficient tests to convince yourself your code’s correctness.
    \end{itemize}
\end{itemize}

\end{document}
