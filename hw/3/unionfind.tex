\documentclass[12pt]{article}

\usepackage{fancyhdr}
\usepackage{fullpage}
\usepackage{hyperref}
\usepackage{microtype}
\usepackage{amsmath,amssymb}
\usepackage[MnSymbol]{mathspec}
\usepackage{mathspec}
\usepackage[final]{listings}

\setlength{\parindent}{0em}
\setlength{\parskip}{0.5em}

\pagestyle{fancy}
\lhead{EECS 214}
\rhead{Fall 2015}
\cfoot{}
\renewcommand{\headrulewidth}{0pt}
\renewcommand{\footrulewidth}{0pt}
\setlength{\headheight}{.25in}
\setlength{\headsep}{.25in}
\addtolength{\topmargin}{-.5in}

\newcommand{\sourcefilelong}{%
  \url{http://goo.gl/12dFgn}%
}

\newcommand{\sourcefile}{%
  \href{http://users.eecs.northwestern.edu/~jesse/course/eecs214-fa15/hw/3/unionfind.rkt}{\texttt{unionfind.rkt}}%
}

\newcommand{\bigO}[1]{%
  \ensuremath{\mathcal O(#1)}%
}

%%%
%%% LISTINGS SETUP
%%%

\lstdefinelanguage{ASL}%
  {sensitive,%
   alsoletter=-+*/\#?,%
   alsodigit=-,%
   morekeywords={%
      and,%
      begin,%
      check-error,%
      check-expect,%
      check-within,%
      cond,%
      define,%
      define-struct,%
      else,%
      if,%
      lambda,%
      local,%
      or,%
   },%
   morekeywords=[2]{%
     cons,%
     \#true,%
     \#false,%
   },%
   morekeywords=[3]{%
     this-is-here-because-suck,%
     cons?,%
     empty?,%
     -,+,*,/,%
     first,%
     rest,%
   }%,
   morecomment=[l];,%
   morecomment=[s]{\#|}{|\#},%
   morestring=[b]"%
  }[keywords,comments,strings]%

\lstnewenvironment{asl}[1][]
{%
  \begingroup
  \lstset{language=ASL,#1}%
}
{%
  \endgroup
}

\lstnewenvironment{cpp}[1][]
{%
  \begingroup
  \lstset{language=C++,#1}%
}
{%
  \endgroup
}


\newcommand{\textcode}{%
  \lstset{language=ASL,basicstyle=\ttfamily\small}%
  \lstinline
}

\lstset{%
  numberstyle=\scriptsize,
  showspaces=false,
  basicstyle=\ttfamily,
  keywordstyle=\ttfamily\bfseries,
  %keywordstyle=[2]\ttfamily\color{blue!70!black},
  %keywordstyle=[3]\ttfamily\color{green!40!black},
  numberstyle=\ttfamily,%\color{blue!70!black},
  commentstyle=\itshape,%\color{violet!80!black},
  escapeinside={($}{$)},
}

\newcommand{\textasl}{%
  \lstset{language=ASL,basicstyle=\ttfamily\small}%
  \lstinline
}

\let\sup^

\catcode`\^\active\relax
\def^#1^{\textasl|#1|}

\begin{document}
\begin{center}
  {\LARGE\textbf{HW3: Union-Find}}
  \par
  {\large \textbf{Due:} Monday, November 23, at 11:59 PM, via Canvas}
\end{center}

\textbf{You may work on your own or with one (1) partner.}
\par\bigskip

For this assignment you will implement the union-find data structure with
path compression and weighted union as we saw in class. Unlike in HW2,
the representation itself is not defined for you, so you’ll have to
define it.

In \sourcefile{} I've supplied headers for the functions that you’ll
need to write, along with two suggested helpers and some code to help
with testing.

\subsection*{Your task}

First you will need to define your representation, the ^UnionFind^ data
type. Each ^UnionFind^ represents a “universe” with a fixed
number of objects identified by natural numbers.

Then you will have to implement five functions:

\begin{tabular}{l@{\texttt{ : }}l@{\hspace{2em}\texttt{;~}}l}
  ^create^    & ^N -> UnionFind^ &            \bigO{n} \\
  ^size^      & ^UnionFind -> N^ &            \bigO{1} \\
  ^union!^    & ^UnionFind N N -> Void^ &     amortized \bigO{\alpha(n)} \\
  ^find^      & ^UnionFind N -> N^ &          amortized \bigO{\alpha(n)} \\
  ^same-set?^ & ^UnionFind N N -> Boolean^ &  amortized \bigO{\alpha(n)} \\
\end{tabular}

{\itshape
(Note: ^N^ is the natural numbers and $\alpha$ is the inverse
\href{https://en.wikipedia.org/wiki/Ackermann_function}{Ackermann
function}.)}

Calling ^(create n)^ returns a new ^UnionFind^ universe (defined by you)
initialized to have ^n^ objects in disjoint singleton sets numbered $0$
to ^n^${} - 1$. Given a universe ^uf^, ^(size uf)^ returns the number of
objects (not sets!) in the universe—that is, ^size^ will always return
the number that was passed to ^create^ when that universe was
initialized.

Functions ^union!^ and ^find^ implement the standard union-find
operations: The function call ^(union! uf n m)^ unions the set
containing ^n^ with the set containing ^m^, if they are not already one
and the same. ^(find uf n)^ returns the representative (root) object
name for the set containing ^n^. The ^find^ function must perform path
compression, and the ^union!^ function must set the parent of the root
of the smaller set to be the root of the larger set. For convenience,
^(same-set? uf n m)^ returns whether objects ^n^ and ^m^ are in the same
set according to union-find universe ^uf^.

\subsection*{Deliverables}

The provided file \sourcefile{} (\sourcefilelong), containing 1) a
definition of your ^UnionFind^ data type, and 2) complete, working
definitions of the five functions specified above. Thorough testing is
strongly recommended but will not be graded.

\end{document}
