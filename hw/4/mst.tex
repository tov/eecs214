\documentclass[12pt]{article}

\usepackage{fancyhdr}
\usepackage{fullpage}
\usepackage{hyperref}
\usepackage{microtype}
\usepackage{amsmath,amssymb}
\usepackage[MnSymbol]{mathspec}
\usepackage{mathspec}
\usepackage[final]{listings}

\setlength{\parindent}{0em}
\setlength{\parskip}{0.5em}

\pagestyle{fancy}
\lhead{EECS 214}
\rhead{Fall 2015}
\cfoot{}
\renewcommand{\headrulewidth}{0pt}
\renewcommand{\footrulewidth}{0pt}
\setlength{\headheight}{.25in}
\setlength{\headsep}{.25in}
\addtolength{\topmargin}{-.5in}

\newcommand{\sourcefilelong}{%
  \url{http://goo.gl/q3QfuL}%
}

\newcommand{\sourcefile}{%
  \href{http://users.eecs.northwestern.edu/~jesse/course/eecs214-fa15/hw/4/mst.rkt}{\texttt{mst.rkt}}%
}

\newcommand{\bigO}[1]{%
  \ensuremath{\mathcal O(#1)}%
}

%%%
%%% LISTINGS SETUP
%%%

\lstdefinelanguage{ASL}%
  {sensitive,%
   alsoletter=-+*/\#?,%
   alsodigit=-,%
   morekeywords={%
      and,%
      begin,%
      check-error,%
      check-expect,%
      check-within,%
      cond,%
      define,%
      define-struct,%
      else,%
      if,%
      lambda,%
      local,%
      or,%
   },%
   morekeywords=[2]{%
     cons,%
     \#true,%
     \#false,%
   },%
   morekeywords=[3]{%
     this-is-here-because-suck,%
     cons?,%
     empty?,%
     -,+,*,/,%
     first,%
     rest,%
   }%,
   morecomment=[l];,%
   morecomment=[s]{\#|}{|\#},%
   morestring=[b]"%
  }[keywords,comments,strings]%

\lstnewenvironment{asl}[1][]
{%
  \begingroup
  \lstset{language=ASL,#1}%
}
{%
  \endgroup
}

\lstnewenvironment{cpp}[1][]
{%
  \begingroup
  \lstset{language=C++,#1}%
}
{%
  \endgroup
}


\newcommand{\textcode}{%
  \lstset{language=ASL,basicstyle=\ttfamily\small}%
  \lstinline
}

\lstset{%
  numberstyle=\scriptsize,
  showspaces=false,
  basicstyle=\ttfamily,
  keywordstyle=\ttfamily\bfseries,
  %keywordstyle=[2]\ttfamily\color{blue!70!black},
  %keywordstyle=[3]\ttfamily\color{green!40!black},
  numberstyle=\ttfamily,%\color{blue!70!black},
  commentstyle=\itshape,%\color{violet!80!black},
  escapeinside={($}{$)},
}

\newcommand{\textasl}{%
  \lstset{language=ASL,basicstyle=\ttfamily\small}%
  \lstinline
}

\let\sup^

\catcode`\^\active\relax
\def^#1^{\textasl|#1|}

\begin{document}
\begin{center}
  {\LARGE\textbf{HW4: Graphs and MSTs}}
  \par
  {\large \textbf{Due:} Monday, December 7, at 11:59 PM, via Canvas}
\end{center}

\textbf{You may work on your own or with one (1) partner.}
\par\bigskip

For this assignment you will implement an API for weighted, undirected
graphs; then you will use this API, along with your solutions to
previous homework assignments, to implement Kruskal’s algorithm for
computing a minimum spanning forest (detailed below).

In \sourcefile{} I've supplied headers for the functions that you’ll
need to write, along with a few suggested helpers and some code to help
with testing.

\subsection*{Background}

\subsubsection*{Definitions}

\begin{itemize}

  \item A graph is \emph{connected} if there is a path from every vertex
    to every other; otherwise it comprises two or more \emph{connected
    components,} each of which is a maximal connected subgraph. (A
    connected component is maximal in the sense that no additional
    vertices could be added and still have it be connected.)

  \item A \emph{spanning tree} of a connected graph $G$ is a subgraph
    that includes all of $G$’s vertices, but only enough edges for it to
    be connected and no more. Cycles would introduce redundant
    connectivity, so it’s a tree. Note that the number of edges in a
    spanning tree is always one less than the number of vertices in the
    original graph.

  \item A \emph{minimum spanning tree} for a connected graph is a
    spanning tree with minimum total weight. (There may be a tie.) We
    can interpret an MST as follows: If vertices represent sites of some
    kind, edges potential connections between them, and weights the
    costs of those edges, then an MST gives the lowest cost way to
    connect all the sites.

\item A graph that isn’t connected has a miniumum spanning tree for each
  of its connected components. This collection of MSTs is a
  \emph{minimum spanning forest.}

\end{itemize}

\subsubsection*{Kruskal’s algorithm}

The result of Kruskal’s algorithm is a graph with the same vertices as
the input graph, but whose edges form a minimum spanning tree (or
forest). The result graph starts with all of the vertices from the input
graph and no edges. In other words, initially each vertex forms its own
(degenerate) connected component.

The algorithm works by maintaining the set of connected components in
the result (using a union-find data structure); it repeatedly adds an
edge that connects two components, thus unifying them into one. In
particular, to achieve minimality, it considers the edges in order from
lightest weight to heaviest. For each edge, if its two vertices are
already in the same connected component of the result graph, the edge is
ignored; but if the edge would connect vertices that are in two
different connected components then the edge is added to the resulting
graph, thus joining the two components into one. When all edges have
been considered then the result is a minimum spanning tree (or forest,
as appropriate).

\subsection*{Your task}

\subsubsection*{Part I: Graphs}

First you will need to define your representation, the ^WUGraph^ data
type. A ^WUGraph^ represents a weighted, unordered graph, where vertices
are identified by consecutive natural numbers from 0, and weights are
arbitrary numbers:

\begin{asl}
;; Vertex is N
;; Weight is Number
\end{asl}

The API also uses a data type for weights that includes infinity:

\begin{asl}
;; A MaybeWeight is one of:
;; -- Weight
;; -- +inf.0
\end{asl}

Your graph representation is up to you---you may use either adjacency
lists or an adjacency matrix.

Once you’ve defined your graph representation, you will have to
implement five functions for working with graphs:

\begin{tabular}{l@{\texttt{ : }}l}
  ^make-graph^   & ^N -> WUGraph^ \\
  ^set-edge!^    & ^WUGraph Vertex Vertex MaybeWeight -> Void^ \\
  ^graph-size^   & ^WUGraph -> N^ \\
  ^get-edge^     & ^WUGraph Vertex Vertex -> MaybeWeight^ \\
  ^get-adjacent^ & ^WUGraph Vertex -> [List-of Vertex]^ \\
\end{tabular}

To construct a graph, we would start with ^(make-graph n)^, which
returns a new graph having ^n^ vertices and no edges. Then we add edges
using ^(set-edge! g u v w)^, which connects vertices ^u^ and ^v^ by an
edge having weight ^w^. The weight ^w^ may be an actual numeric weight
or it may be ^+inf.0^, which effectively removes the edge.

^(graph-size g)^ returns the number of vertices in ^g^, which is the
same as the number originally passed to ^make-graph^ to create the
graph. ^(get-edge g u v)^ returns the weight of the edge from ^u^ to
^v^, which will be ^+inf.0^ if there is no such edge. Note that because
the graph is undirected, ^(get-edge g u v)^ should always be the same as
^(get-edge g v u)^.\footnote{This probably means that ^set-edge!^ needs to
maintain an invariant.} Finally ^(get-adjacent g v)^ returns a list of all
the vertices adjacent to ^v^ in graph ^g^---note that an undirected
graph does not distinguish predecessors from successors.

\subsubsection*{Part II: MSTs}

Once you have a working graph API, you should implement Kruskal’s
algorithm as a function ^kruskal-mst : WUGraph -> WUGraph^. Given any
weighted, undirected graph ^g^, ^(kruskal-mst g)^ returns a graph with
the same vertices as ^g^ and edges forming a minimum spanning forest,
using the algorithm as described above.

In order to maintain and query the set of connected components,
Kruskal’s algorithm uses union-find. You should use your union-find data
structure and operations from HW3. There’s no good way to import it, so
you will have to copy and paste. (If you’re working with a different
partner now than you did for HW3 then you may use either your own
union-find or theirs.)

In order to consider the edges in order by increasing weight, Kruskal’s
algorithm requires sorting the edges by weight. I used my HW2 solution
to write a heap sort (which works by adding all the things to sort to a
heap and then removing them), but you may use any sorting algorithm you
wish.

I’ve listed some helpers that you may find useful at the bottom of
\sourcefile.

\subsection*{Deliverable}

The provided file \sourcefile{} (\sourcefilelong), containing

\begin{enumerate} \item a definition of your ^WUGraph^ data type, \item
    complete, working definitions of the five graph API functions
  specified above, and \item a working implementation of ^kruskal-mst^.
\end{enumerate}

Thorough testing is strongly recommended but will not be graded.

\end{document}
