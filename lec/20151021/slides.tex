\documentclass{beamer}

\RequirePackage{ucs}
\usepackage[quiet]{fontspec}
\usepackage{xunicode}
\usepackage{xltxtra}
\usepackage{graphicx}
%\usepackage{forloop}
\usepackage{stmaryrd}
%\usepackage{mathtools}
\usepackage{fancyvrb}
\usepackage{amsmath,amssymb,amsthm}

\usepackage{tikz}
\usetikzlibrary{arrows.meta}
\usetikzlibrary{backgrounds}
\usetikzlibrary{calc}
\usetikzlibrary{datavisualization}
\usetikzlibrary{datavisualization.formats.functions}
\usetikzlibrary{fit}
\usetikzlibrary{positioning}
\usetikzlibrary{scopes}


\newcommand\bigO[1]{\ensuremath{\mathcal{O}(#1)}}
\newcommand\fun[1]{\ensuremath{\lambda #1.\,}}
\renewcommand\T[1]{\ensuremath{T_{\mathit{#1}}}}

%%%
%%% BASIC SETTINGS
%%%

\frenchspacing
\unitlength=0.01\textwidth
\thicklines
\urlstyle{sf}
\graphicspath{{images/}}

\defaultfontfeatures{
    Mapping=tex-text,
    Scale=MatchLowercase,
}

% \setmainfont{Trattatello}
\setmainfont{TeX Gyre Schola}
\setmonofont{Menlo}

%%%
%%% VERBATIM SETUP
%%%

\DefineVerbatimEnvironment
  {code}{Verbatim}
  {% commandchars=\\\‹\›,
   commentchar=\%,
   codes={\catcode`$=3\relax},
   baselinestretch=.8,
   formatcom=\relax}
\renewcommand\FancyVerbFormatLine[1]{#1\strut}
\fvset{% commandchars=\\\‹\›,
       codes={\catcode`$=3\relax},
       formatcom=\relax}

\let\super^\relax % save old ^
\DefineShortVerb{\^}

%%%
%%% BEAMER SETUP
%%%

\setbeamercolor{frametitle}{fg=black}
\setbeamercolor{normal text}{fg=black}

\usefonttheme{serif}

\setbeamertemplate{frametitle}
  {\begin{centering}\medskip
   \insertframetitle\par
   \end{centering}}

\setbeamertemplate{itemize item}{$\bullet$}

\setbeamertemplate{navigation symbols}{}
\setbeamertemplate{footline}[text line]{%
    \hfill\strut{%
        \small\color{black!75}%
        \texttt{\insertframenumber:\insertframeslidenumber}%
    }%
    \hfill%
}

\makeatletter
\newcount\frameslidetemp
\newcommand\insertframeslidenumber{%
  \frameslidetemp\c@page\relax
  \advance\frameslidetemp-\beamer@startpageofframe\relax
  \advance\frameslidetemp1\relax
  \the\frameslidetemp
}
\makeatother

%%%
%%% GENERALLY USEFUL LITTLE MACROS
%%%

% For using \alt within verbatim:
\newcommand\ALT[3]{\alt<#1>{#2}{#3}}

% Or: \ALT = \S\alt
\def\S#1#2{#2<#1>}

\newcommand<>\always[1]{#1}

%%%
%%% COLORS
%%%

\newcommand\K[1]{\textcolor{orange!70!black}{#1}}
\newcommand\REF{\textcolor{green!50!black}{ref}}

%%%
%%% TIKZ
%%%

% \place[STYLE]{LOCATION}{CONTENT} places CONTENT at LOCATION, applying
% STYLE
\newcommand\place[3][]{%
  \tikz[orp] \node[#1] at (#2) {#3};%
}

% \anchor{NAME} creates an addressable TikZ coordinate at the current
% location.
\newcommand\anchor[1]{%
  \tikz[orp] \coordinate (#1);%
}

% \nil{NODE} draws a slash across the node; if NODE is a square then we
% get the standard graphic notation for null/nil.
\newcommand\nil[1][1em]{%
  \begin{tikzpicture}[nil]
    \node[pointer tail, line width=0, opacity=0] {};
    \draw[overlay]
      node[pointer tail] (dummy) {}
      (dummy.west |- dummy.south) -- (dummy.east |- dummy.north);
  \end{tikzpicture}%
}

\makeatletter
  \newcommand\cons[1][]{\cons@kont{#1}}
  \def\cons@kont#1#2(#3)#4#5{\struct@kont{#1}{}(#3){#4&#5}}

  \newcommand\struct[1][]{\struct@kont{#1}}
  \def\struct@kont#1#2(#3)#4{
      \node[draw, inner sep=0, #1] (#3) {%
        \begin{tabular}{|*{100}{c|}}
          \hline
          #4\\
          \hline
        \end{tabular}
      }
  }
\makeatother

% \genpointer[STYLE]{BEND}{TARGET}
\newcommand\genpointer[2][genpointer]{%
  \tikz[remember picture, #1]
    \draw
      node[pointer tail] {}
      edge[->, overlay] (#2) ;%
}

% \hpointer[BEND]{TARGET} for a highlighted pointer
\newcommand\hpointer[2][]{%
  \genpointer[highlit pointer,#1]{#2}%
}

% \ppointer[BEND]{TARGET} for a non-highlighted pointer
\newcommand\ppointer[2][]{%
  \genpointer[pointer,#1]{#2}%
}

% \pointer[BEND]{TARGET} is highlighted if used in a highlighted context
\newcommand\pointer[2][]{%
  \ifhighlit\let\pointertemp\hpointer\else\let\pointertemp\ppointer\fi
  \pointertemp[#1]{#2}%
}

\newcommand\genpointerbase[2][genpointer]{%
  \tikz[remember picture, #1]
    \draw node[pointer tail] (#2) {};%
}

\newcommand\ppointerbase[2][genpointer]{%
  \genpointerbase[pointer,#1]{#2}%
}

\newcommand\hpointerbase[2][genpointer]{%
  \genpointerbase[highlit pointer,#1]{#2}%
}

\newcommand\pointerbase[2][]{%
  \ifhighlit\let\pointertemp\hpointerbase\else\let\pointertemp\ppointerbase\fi
  \pointertemp[#1]{#2}%
}

\newcommand\genpointerpoint[3][genpointer]{
  \draw (#2) edge[->, remember picture, overlay, #1] (#3) ;
}

\newcommand\ppointerpoint[2][genpointer]{%
  \genpointerpoint[pointer,#1]{#2}%
}

\newcommand\hpointerpoint[2][genpointer]{%
  \genpointerpoint[highlit pointer,#1]{#2}%
}

\newcommand\pointerpoint[2][]{%
  \ifhighlit\let\pointertemp\hpointerpoint\else\let\pointertemp\ppointerpoint\fi
  \pointertemp[#1]{#2}%
}

\tikzset{
  orp/.style={
    overlay,
    remember picture,
  },
  rem/.style={
    remember picture,
  },
  every node/.style={
    very thick,
    inner sep=3pt,
  },
  every path/.style={very thick},
  ref/.style={
    draw=red,
    densely dotted,
    font=\ttfamily,
  },
  cons/.style={
    draw=black,
    font=\ttfamily,
  },
  genpointer/.style={
    solid,
    bend left=30,
    >=Latex,
  },
  pointer/.style={
    genpointer,
    color=blue!50!black,
    opacity=.5,
  },
  highlit pointer/.style={
    genpointer,
    color=red!70!black,
  },
  nil/.style={
    color=black!50,
  },
  pointer tail/.style={
    draw,
    circle,
    text width=0,
    text height=1ex,
    inner sep=0,
  },
  highlight/.style={
    color=yellow!50!white!90!black,
    thick,
    draw,
    draw opacity=0.5,
    outer color=yellow!50,
    inner color=yellow!30,
    rounded corners=2pt,
  },
}

\newif\ifhighlit
\highlitfalse

% \highlight<WHEN>[NAME]{CONTENT}
\newcommand<>\highlight[2][\relax]{%
  \ifx#1\relax
    \begin{tikzpicture}
  \else
    \begin{tikzpicture}[remember picture]
  \fi
    \node[
      inner sep=0pt,
      outer sep=0pt,
      text depth=0pt,
      align=left,
    ] (highlight temp) {%
        \only#3{\highlittrue}%
        #2%
        \only#3{\highlitfalse}%
    };
    \begin{scope}[on background layer]
      \coordinate (highlight temp 1) at
        ($(highlight temp.north east) + (2pt,2pt)$);
      \coordinate (highlight temp 2) at
        ($(highlight temp.base west) + (-2pt,-3pt)$);
      \node#3 [
        overlay,
        highlight,
        inner sep=0pt,
        fit=(highlight temp 1) (highlight temp 2),
      ] {};
    \end{scope}
    \ifx#1\relax
      \relax
    \else
      \node (#1) [
        overlay,
        inner sep=0pt,
        fit=(highlight temp 1) (highlight temp 2),
      ] {};
    \fi
  \end{tikzpicture}%
}

\newcommand\HI[2]{\highlight<#1>{#2}}

% \huncover[NAME]{STEP}{CONTENT} uncovers from STEP onward, highlighting
% at STEP but not thereafter.
\newcommand\huncover[3][\relax]{%
  \uncover<#2->{\highlight<#2>[#1]{#3}}%
}

% \huncover[NAME]{STEP}{CONTENT} onlys from STEP onward, highlighting
% at STEP but not thereafter.
\newcommand\honly[3][\relax]{%
  \only<#2->{\highlight<#2>[#1]{#3}}%
}

\begin{document}

%%%
%%% TITLE PAGE
%%%

\begin{frame}[fragile]\relax
  \thispagestyle{empty}

  %% Need to initialize Tikz opacity inside a frame, not outside:
  \begin{tikzpicture}[opacity=0,overlay]\end{tikzpicture}

  \tikz[overlay, remember picture]
  \node at ($ (current page.south west) + (-1.5ex, -1ex) $) [anchor=south west]
  {\begin{tikzpicture}
    \datavisualization[
      xy Cartesian,
      visualize as smooth line,
      data/format=function,
      visualize as smooth line/.list={
        log
        ,linear
        ,linlog
        ,quadratic
        ,exp
      },
      style sheet=strong colors,
      every visualizer/.style={
        line width=1ex,
        opacity=0.3,
      },
      x axis={unit length=\paperwidth per 100 units},
      y axis={unit length=\paperheight per 100 units},
     ]
    data [set=log] {
      var x : interval [1:100];
      func y = 10*ln(\value x);
    }
    data [set=linear] {
      var x : interval [0:100];
      func y = \value x;
    }
    data [set=linlog] {
      var x : interval [1:100];
      func y = 0.2 * \value x*ln(\value x);
    }
     data [set=quadratic] {
       var x : interval [0:100];
       func y = 0.03*\value x*\value x;
     }
     data [set=exp] {
       var x : interval [0:70];
       func y = exp(\value x/15);
     }
    ;
  \end{tikzpicture}
  };

  \begin{center}
  {\Huge Asymptotic Complexity}

  \bigskip
  \Large
  EECS 214

  \medskip
  October 21, 2015
  \end{center}

\end{frame}

\begin{frame}[fragile]{Two kinds of sequences}
  Doubly-linked list: \medskip

  \begin{tikzpicture}[
    remember picture,
    every node/.style={anchor=west},
  ]
    \struct[at={(0, 2)}] (a) {\nil & a & \pointerbase{a next}};
    \struct[at={(2.4, 2)}] (b) {\pointer{a} & b & \pointerbase{b next}};
    \struct[at={(4.8, 2)}] (c) {\pointer{b} & c & \pointerbase{c next}};
    \struct[at={(7.2, 2)}] (d) {\pointer{c} & d & \nil};
    \pointerpoint{a next}{b}
    \pointerpoint{b next}{c}
    \pointerpoint{c next}{d}
  \end{tikzpicture}

  \bigskip

  Array with size: \medskip

  \begin{tikzpicture}[
    remember picture,
    every node/.style={anchor=west},
  ]
    \struct[at={(2, 0)}] (array) { a & b & c & d & ? & ? & ? & ? };
    \cons[at={(0, 0)}] (head) {4} {\pointer{array.north west}};
  \end{tikzpicture}
\end{frame}

\begin{frame}[fragile,c]{Representing sets}
  \strut\kern-2em\begin{tabular}{l|cccc}
    & list & array & \only<5->{sorted list & sorted array} \\
    \hline
    empty? & $c$ & $c$ & \uncover<6->{$c$ & \uncover<7->{$c$}}\\
    \uncover<2->{find & $c_1N + c_2$ & $c_1N + c_2$ & \uncover<6->{$c_1N
    + c_2$ & \uncover<8->{$c_1\log N + c_2$}}} \\
    \uncover<3->{insert & $c_1N + c_2$ & $c_1N + c_2$ &
    \uncover<6->{$c_1N + c_2$ & \uncover<9->{$c_1N + c_2$}}}\\
    \uncover<4->{remove & $c_1N + c_2$ & $c_1N + c_2$ &
    \uncover<6->{$c_1N + c_2$ & \uncover<10->{$c_1N + c_2$}}}\\
    \uncover<11->{%
      union &
       \uncover<12->{%
       $c_1N\super2 + c_2N + c_3$ &
       \uncover<13->{$c_1N\super2 + c_2N + c_3$ &
       \uncover<14->{$c_1N + c_2$ &
       \uncover<15->{$c_1N + c_2$}}}}
    } \\
  \end{tabular}
\end{frame}

\end{document}
