\RequirePackage[l2tabu, orthodox]{nag}

\documentclass{beamer}

\usepackage{microtype}
\usepackage{amsmath,amssymb}
%??? :( \usepackage[MnSymbol]{mathspec}
\usepackage{mathspec}
\usepackage[final]{listings}
\usepackage{graphicx}

\setallmainfonts{TeX Gyre Schola}
\setallmonofonts{Menlo}

%\usepackage{xunicode}
%\usepackage{xltxtra}
%\usepackage{graphicx}
%\usepackage{forloop}
%\usepackage{stmaryrd}
%\usepackage{mathtools}
%\usepackage{fancyvrb}

\usepackage{tikz}
% \usetikzlibrary{arrows.meta}
\usetikzlibrary{backgrounds}
\usetikzlibrary{calc}
% \usetikzlibrary{datavisualization}
% \usetikzlibrary{datavisualization.formats.functions}
\usetikzlibrary{fit}
% \usetikzlibrary{positioning}
% \usetikzlibrary{scopes}

\def\bigO#1{\ensuremath{\mathcal O(#1)}}
\newcommand\fun[1]{\ensuremath{\lambda #1.\,}}
\renewcommand\T[1]{\ensuremath{T_{\mathit{#1}}}}

%%%
%%% BASIC SETTINGS
%%%

\frenchspacing
\unitlength=0.01\textwidth
\thicklines
\urlstyle{sf}
\graphicspath{{images/}}

\defaultfontfeatures{
    Mapping=tex-text,
    Scale=MatchLowercase,
}

%%%
%%% LISTINGS SETUP
%%%

\lstdefinelanguage{ASL}%
  {sensitive,%
   alsoletter=-+*/\#?,%
   alsodigit=-,%
   morekeywords={%
      and,%
      begin,%
      check-error,%
      check-expect,%
      check-within,%
      cond,%
      define,%
      define-struct,%
      else,%
      if,%
      lambda,%
      local,%
      or,%
   },%
   morekeywords=[2]{%
     cons,%
     \#true,%
     \#false,%
   },%
   morekeywords=[3]{%
     this-is-here-because-suck,%
     cons?,%
     empty?,%
     -,+,*,/,%
     first,%
     rest,%
   }%,
   morecomment=[l];,%
   morecomment=[s]{\#|}{|\#},%
   morestring=[b]"%
  }[keywords,comments,strings]%

\lstnewenvironment{asl}[1][]
{%
  \begingroup
  \lstset{language=ASL,#1}%
}
{%
  \endgroup
}

\lstnewenvironment{cpp}[1][]
{%
  \begingroup
  \lstset{language=C++,#1}%
}
{%
  \endgroup
}


\newcommand{\textcode}{%
  \lstset{language=ASL,basicstyle=\ttfamily\small}%
  \lstinline
}

\lstset{%
  numberstyle=\scriptsize,
  showspaces=false,
  basicstyle=\ttfamily,
  keywordstyle=\ttfamily\color{red!70!black},
  keywordstyle=[2]\ttfamily\color{blue!70!black},
  keywordstyle=[3]\ttfamily\color{green!40!black},
  numberstyle=\ttfamily\color{blue!70!black},
  commentstyle=\itshape\color{violet!80!black},
  escapeinside={($}{$)},
}

\newcommand{\textasl}{%
  \lstset{language=ASL,basicstyle=\ttfamily\small}%
  \lstinline
}

\let\sup^

\catcode`\^\active\relax
\def^#1^{\textasl|#1|}

%%%
%%% BEAMER SETUP
%%%

\setbeamercolor{frametitle}{fg=black}
\setbeamercolor{normal text}{fg=black}

\usefonttheme{serif}

\setbeamertemplate{frametitle}
  {\begin{center}\medskip
   \insertframetitle\par
   \end{center}}

\setbeamertemplate{itemize item}{$\bullet$}

\setbeamertemplate{navigation symbols}{}
\setbeamertemplate{footline}[text line]{%
    \hfill\strut{%
        \small\color{black!75}%
        \texttt{\insertframenumber:\insertframeslidenumber}%
    }%
    \hfill%
}

\makeatletter
\newcount\frameslidetemp
\newcommand\insertframeslidenumber{%
  \frameslidetemp\c@page\relax
  \advance\frameslidetemp-\beamer@startpageofframe\relax
  \advance\frameslidetemp1\relax
  \the\frameslidetemp
}
\makeatother

%%%
%%% GENERALLY USEFUL LITTLE MACROS
%%%

% For using \alt within verbatim:
\newcommand\ALT[3]{\alt<#1>{#2}{#3}}

% Or: \ALT = \S\alt
\def\S#1#2{#2<#1>}

\newcommand<>\always[1]{#1}

\newcommand\widen[1][1.2]{%
    \setlength\baselineskip{#1\baselineskip}%
}

%%%
%%% COLORS
%%%

\newcommand\K[1]{\textcolor{orange!70!black}{#1}}
\newcommand\REF{\textcolor{green!50!black}{ref}}

%%%
%%% TIKZ
%%%

% \place[STYLE]{LOCATION}{CONTENT} places CONTENT at LOCATION, applying
% STYLE
\newcommand\place[3][]{%
  \tikz[orp] \node[#1] at (#2) {#3};%
}

% \anchor{NAME} creates an addressable TikZ coordinate at the current
% location.
\newcommand\anchor[1]{%
  \tikz[orp] \coordinate (#1);%
}

% \nil{NODE} draws a slash across the node; if NODE is a square then we
% get the standard graphic notation for null/nil.
\newcommand\nil[1][1em]{%
  \begin{tikzpicture}[nil]
    \node[pointer tail, line width=0, opacity=0] {};
    \draw[overlay]
      node[pointer tail] (dummy) {}
      (dummy.west |- dummy.south) -- (dummy.east |- dummy.north);
  \end{tikzpicture}%
}

\makeatletter
  \newcommand\cons[1][]{\cons@kont{#1}}
  \def\cons@kont#1#2(#3)#4#5{\struct@kont{#1}{}(#3){#4&#5}}

  \newcommand\struct[1][]{\struct@kont{#1}}
  \def\struct@kont#1#2(#3)#4{
      \node[draw, inner sep=0, #1] (#3) {%
        \begin{tabular}{|*{100}{c|}}
          \hline
          #4\\
          \hline
        \end{tabular}
      }
  }
\makeatother

% \genpointer[STYLE]{BEND}{TARGET}
\newcommand\genpointer[2][genpointer]{%
  \tikz[remember picture, #1]
    \draw
      node[pointer tail] {}
      edge[->, overlay] (#2) ;%
}

% \hpointer[BEND]{TARGET} for a highlighted pointer
\newcommand\hpointer[2][]{%
  \genpointer[highlit pointer,#1]{#2}%
}

% \ppointer[BEND]{TARGET} for a non-highlighted pointer
\newcommand\ppointer[2][]{%
  \genpointer[pointer,#1]{#2}%
}

% \pointer[BEND]{TARGET} is highlighted if used in a highlighted context
\newcommand\pointer[2][]{%
  \ifhighlit\let\pointertemp\hpointer\else\let\pointertemp\ppointer\fi
  \pointertemp[#1]{#2}%
}

\newcommand\genpointerbase[2][genpointer]{%
  \tikz[remember picture, #1]
    \draw node[pointer tail] (#2) {};%
}

\newcommand\ppointerbase[2][genpointer]{%
  \genpointerbase[pointer,#1]{#2}%
}

\newcommand\hpointerbase[2][genpointer]{%
  \genpointerbase[highlit pointer,#1]{#2}%
}

\newcommand\pointerbase[2][]{%
  \ifhighlit\let\pointertemp\hpointerbase\else\let\pointertemp\ppointerbase\fi
  \pointertemp[#1]{#2}%
}

\newcommand\genpointerpoint[3][genpointer]{
  \draw (#2) edge[->, remember picture, overlay, #1] (#3) ;
}

\newcommand\ppointerpoint[2][genpointer]{%
  \genpointerpoint[pointer,#1]{#2}%
}

\newcommand\hpointerpoint[2][genpointer]{%
  \genpointerpoint[highlit pointer,#1]{#2}%
}

\newcommand\pointerpoint[2][]{%
  \ifhighlit\let\pointertemp\hpointerpoint\else\let\pointertemp\ppointerpoint\fi
  \pointertemp[#1]{#2}%
}

\tikzset{
  onslide/.code args={<#1>#2}{%
    \only<#1>{\pgfkeysalso{#2}}
  },
  orp/.style={
    overlay,
    remember picture,
  },
  rem/.style={
    remember picture,
  },
  every node/.style={
    very thick,
    inner sep=3pt,
  },
  every path/.style={very thick},
  ref/.style={
    draw=red,
    densely dotted,
    font=\ttfamily,
  },
  cons/.style={
    draw=black,
    font=\ttfamily,
  },
  genpointer/.style={
    solid,
    bend left=30,
    >=Latex,
  },
  pointer/.style={
    genpointer,
    color=blue!50!black,
    opacity=.5,
  },
  highlit pointer/.style={
    genpointer,
    color=red!70!black,
  },
  nil/.style={
    color=black!50,
  },
  pointer tail/.style={
    draw,
    circle,
    text width=0,
    text height=1ex,
    inner sep=0,
  },
  highlight base/.style={
    outer color=#1!50,
    inner color=#1!30,
  },
  highlight base/.default=yellow,
  highlight/.style={
    highlight base,
    color=yellow!50!white!90!black,
    draw opacity=0.5,
    thick,
    draw,
    rounded corners=2pt,
  },
  bintree/.style={
    level 1/.style={sibling distance=#1},
    level 2/.style={sibling distance=#1/2},
    level 3/.style={sibling distance=#1/4},
    level 4/.style={sibling distance=#1/8},
    level 5/.style={sibling distance=#1/16},
  },
  bintree/.default={8em},
}

\newif\ifhighlit
\highlitfalse

% \highlight<WHEN>[NAME]{CONTENT}
\newcommand<>\highlight[2][\relax]{%
  \ifx#1\relax
    \begin{tikzpicture}
  \else
    \begin{tikzpicture}[remember picture]
  \fi
    \node[
      inner sep=0pt,
      outer sep=0pt,
      text depth=0pt,
      align=left,
    ] (highlight temp) {%
        \only#3{\highlittrue}%
        #2%
        \only#3{\highlitfalse}%
    };
    \begin{scope}[on background layer]
      \coordinate (highlight temp 1) at
        ($(highlight temp.north east) + (2pt,2pt)$);
      \coordinate (highlight temp 2) at
        ($(highlight temp.base west) + (-2pt,-3pt)$);
      \node#3 [
        overlay,
        highlight,
        inner sep=0pt,
        fit=(highlight temp 1) (highlight temp 2),
      ] {};
    \end{scope}
    \ifx#1\relax
      \relax
    \else
      \node (#1) [
        overlay,
        inner sep=0pt,
        fit=(highlight temp 1) (highlight temp 2),
      ] {};
    \fi
  \end{tikzpicture}%
}

\newcommand\HI[2]{\highlight<#1>{#2}}

% \huncover[NAME]{STEP}{CONTENT} uncovers from STEP onward, highlighting
% at STEP but not thereafter.
\newcommand\huncover[3][\relax]{%
  \uncover<#2->{\highlight<#2>[#1]{#3}}%
}

% \huncover[NAME]{STEP}{CONTENT} onlys from STEP onward, highlighting
% at STEP but not thereafter.
\newcommand\honly[3][\relax]{%
  \only<#2->{\highlight<#2>[#1]{#3}}%
}

\begin{document}

%%%
%%% TITLE PAGE
%%%

\newcommand{\takeaways}{
  \begin{frame}{Take-aways}
    \begin{itemize}
      \item What is a priority queue is all about?
      \item How is the \emph{heap property} defined?
      \item What does a binary heap look like?
      \item How do its operations work?
      \item What are their time complexities?
    \end{itemize}
  \end{frame}
}

\newcommand{\FN}[1]{\ensuremath{\mathop{\mathit{#1}}}}
\newcommand{\PrioQ}{\ensuremath{\mathsf{PrioQ}}}
\newcommand{\Element}{\ensuremath{\mathsf{Element}}}
\newcommand{\Bool}{\ensuremath{\mathsf{Bool}}}

\begin{frame}
  \thispagestyle{empty}
  \begin{tikzpicture}[overlay, remember picture]
    \path
    (current page.south west) +(-.8em, -1em)
    node[anchor=south west]
    {\includegraphics{heap.jpg}}
    ;
  \end{tikzpicture}
\end{frame}

\begin{frame}[fragile]\relax
  \thispagestyle{empty}

  %% Need to initialize Tikz opacity inside a frame, not outside:
  \begin{tikzpicture}[overlay, remember picture, opacity=.25]
    \path
    (current page.south west) +(-.8em, -1em)
    node[anchor=south west]
    {\includegraphics{heap.jpg}}
    ;
  \end{tikzpicture}

  \begin{center}
  {\Huge Heaps}

  \bigskip
  \Large
  EECS 214

  \medskip
  November 4, 2015
  \end{center}
\end{frame}

\takeaways

\begin{frame}{The \emph{priority queue} ADT}
    \def\2{\uncover<2->}
    \def\3{\uncover<3->}
    \begin{center}
  \begin{tabular}{l|cc}
      \2{
      \hfill representation & linked \3{& sorted\only<5->{ ring}} \\
          operation & list \3{& \alt<5->{buffer}{array}} \\
          \hline
      }%
      $\FN{Empty}() : \PrioQ$ \2{& \bigO{1}} \3{& \bigO{1}} \\[4pt]
      $\FN{Empty?}(\PrioQ) : \Bool$ \2{& \bigO{1}} \3{& \bigO{1}}
      \\[4pt]
      $\FN{Insert}(\PrioQ, \Element)$ \2{& \bigO{1}} \3{& \bigO{n}}
      \\[4pt]
      $\FN{FindMin}(\PrioQ) : \Element$ \2{& \bigO{n}} \3{& \bigO{1}}
      \\[4pt]
      $\FN{RemoveMin}(\PrioQ)$ \2{& \bigO{n}} \3{&
          \highlight<4-5>{\alt<5->{\bigO{1}}{\bigO{n}}}} 
  \end{tabular}
    \end{center}
  \bigskip
  Note\only<2->{s}:
  \begin{enumerate}
      \item[\uncover<2->{1.}]
          An $\Element$ has a \emph{key}; keys are totally ordered
      \item<2->[2.]
          $n$ is the number of elements
  \end{enumerate}
\end{frame}

\begin{frame}{We can do better}
    \begin{center}
    \Large\widen
    A \emph{\only<3>{min-}\only<4>{max-}heap} is a tree that \\
    satisfies the \emph{\only<3>{min-}\only<4>{max-}heap property}\uncover<2->{: \\
    every element’s key is \highlight<3->{\alt<4>{greater}{less} than} \\
    all of its descendants’ keys}
    \end{center}
\end{frame}

\begin{frame}{Heaps versus search trees}
  \begin{tabbing}
      \textbf{min-heap property:} \\[.5em]
    \qquad\=\+
    for all nodes $n$, \\
     \qquad\=\+•\quad$n.\FN{key} < n.\FN{left.key}$, and \\
     •\quad$n.\FN{key} < n.\FN{right.key}$ \-\- \\[1em]
     \textbf{BST property:} \\[.5em]
    \qquad\=\+
    for all nodes $n$, \\
     \qquad\=\+•\quad \=\+for all of $n$’s left-descendants $\ell$, \\
     $\ell.\FN{key} < n.\FN{key}$, and \-\\
     •\quad \=\+for all of $n$’s right-descendants $r$, \\
     $r.\FN{key} > n.\FN{key}$
\end{tabbing}
\end{frame}

\begin{frame}[fragile]{Definition: \emph{complete tree}}
  A tree is \emph{complete} if the levels are all filled in
  left-to-right

  \pause
  \medskip
  Like this:

  \medskip
  \strut\hfill
  \begin{tikzpicture}
      [
        every node/.style={circle,draw=black},
      ]
      \node (n0) at (0,0) {};
      \pause
      \node (n1) at (-2,-1) {} edge (n0);
      \pause
      \node (n2) at (2,-1) {} edge (n0);
      \pause
      \node (n3) at (-3,-2) {} edge (n1);
      \pause
      \node (n4) at (-1,-2) {} edge (n1);
      \pause
      \node (n5) at (1,-2) {} edge (n2);
      \pause
      \node (n6) at (3,-2) {} edge (n2);
      \pause
      \node (n7) at (-3.5,-3) {} edge (n3);
      \pause
      \node (n8) at (-2.5,-3) {} edge (n3);
      \pause
      \node (n9) at (-1.5,-3) {} edge (n4);
      \pause
      \node (n10) at (-0.5,-3) {} edge (n4);
      \pause
      \node (n11) at (0.5,-3) {} edge (n5);
      \pause
      \node (n12) at (1.5,-3) {} edge (n5);
      \pause
      \node (n13) at (2.5,-3) {} edge (n6);
      \pause
      \node (n14) at (3.5,-3) {} edge (n6);
  \end{tikzpicture}
  \hfill\strut
\end{frame}

\begin{frame}[fragile]{Definition: \emph{binary heap}}
  \centering
  A \emph{binary heap} is a complete binary tree \\ satisfying the heap property

  \pause
  \medskip
  Like this:

  \medskip
  \strut\hfill
  \begin{tikzpicture}
      [
        every node/.style={
          circle,
          draw=black,
          inner sep=2pt,
        },
      ]
      \node (n0) at (0,0) {3};
      \pause
      \node (n1) at (-2,-1) {5} edge (n0);
      \pause
      \node (n2) at (2,-1) {8} edge (n0);
      \pause
      \node (n3) at (-3,-2) {17} edge (n1);
      \pause
      \node (n4) at (-1,-2) {6} edge (n1);
      \pause
      \node (n5) at (1,-2) {9} edge (n2);
      \pause
      \node (n6) at (3,-2) {60} edge (n2);
      \pause
      \node (n7) at (-3.5,-3) {20} edge (n3);
      \pause
      \node (n8) at (-2.5,-3) {37} edge (n3);
      \pause
      \node (n9) at (-1.5,-3) {44} edge (n4);
      \pause
      \node (n10) at (-0.5,-3) {7} edge (n4);
      \pause
      \node (n11) at (0.5,-3) {12} edge (n5);
      \pause
      \node (n12) at (1.5,-3) {45} edge (n5);
      \pause
      \node (n13) at (2.5,-3) {87} edge (n6);
      \pause
      \node (n14) at (3.5,-3) {62} edge (n6);
  \end{tikzpicture}
  \hfill\strut
\end{frame}

\begin{frame}[fragile]{Operation \FN{FindMin}}
  This one is easy:

  \begin{center}
  \bigskip
  \strut\hfill
  \begin{tikzpicture}
      [
        every node/.style={
          circle,
          draw=black,
          inner sep=2pt,
        },
      ]
      \node[onslide=<2->{highlight base}] (n0) at (0,0) {3};
      \node (n1) at (-2,-1) {5} edge (n0);
      \node (n2) at (2,-1) {8} edge (n0);
      \node (n3) at (-3,-2) {17} edge (n1);
      \node (n4) at (-1,-2) {6} edge (n1);
      \node (n5) at (1,-2) {10} edge (n2);
      \node (n6) at (3,-2) {60} edge (n2);
      \node (n7) at (-3.5,-3) {20} edge (n3);
      \node (n8) at (-2.5,-3) {37} edge (n3);
      \node (n9) at (-1.5,-3) {44} edge (n4);
      \node (n10) at (-0.5,-3) {14} edge (n4);
      \node (n11) at (0.5,-3) {12} edge (n5);
  \end{tikzpicture}
  \hfill\strut
  \end{center}

  \bigskip
  \uncover<3->{
    How long does this take?
    \LARGE\quad
    \huncover{4}{\bigO{1}}%
  }
\end{frame}

\begin{frame}[fragile]{Operation \FN{Insert}}
  \begin{overprint}
    \onslide<1-4>
      This one’s a bit harder. Let’s insert 11 into the heap.

  \smallskip
  \uncover<2->{Step 1: Add it at the end of the heap}

  \uncover<3->{Step 2: Check if the heap condition is (locally!)
  preserved}

  \smallskip
  \uncover<4->{It is, so we’re done! Why is the local check
  sufficient?}

  \onslide<5-9>
  Okay, let’s try inserting 9 instead.
  \par\smallskip
  \uncover<6->{
    The local invariant is broken! How can we fix it?
  }
  \par
  \uncover<7->{
    Swap the troublesome node with its parent.
  }
  \par
  \uncover<8->{
    Now we check 9's new parent. \uncover<9->{Looks good.}
  }
  \onslide<10-15>
    Okay, now let’s insert 2.
    \par\smallskip
    \uncover<11->{Check the local invariant. \uncover<12->{It’s broken!}}
    \par
    \uncover<13->{So swap with the parent. Still broken!}
    \par
    \uncover<14->{So “bubble up” until the invariant is restored.}
  \onslide<16->
    When bubbling up, why didn’t we compare the node against its other
    child? (\emph{E.g.}\/: comparing 2 to 5)
    \par\smallskip
    \uncover<17->{
      Before swap, node $<$ parent, but also parent
      $<$ other child (by heap condition).
      Transitivity of $<$ tells us that node $<$ other child!
    }
  \end{overprint}

  \begin{center}
  \strut\hfill
  \begin{tikzpicture}
      [
        every node/.style={
          circle,
          draw=black,
          inner sep=2pt,
        },
      ]
      \node[onslide=<15>{highlight base}]
            (n0) at (0,0) {\alt<15->{2}{3}};
      \node (n1) at (-2,-1) {5} edge (n0);
      \node[onslide=<14>{highlight base}]
           (n2) at (2,-1) {\alt<15->{3}{\alt<14>{2}{8}}} edge (n0);
      \node (n3) at (-3,-2) {17} edge (n1);
      \node (n4) at (-1,-2) {6} edge (n1);
      \node[onslide=<7-9>{highlight base}]
          (n5) at (1,-2) {\alt<7->{9}{10}} edge (n2);
      \node[onslide=<13>{highlight base}]
            (n6) at (3,-2) {\alt<14->{8}{\alt<13>{2}{60}}} edge (n2);
      \node (n7) at (-3.5,-3) {20} edge (n3);
      \node (n8) at (-2.5,-3) {37} edge (n3);
      \node (n9) at (-1.5,-3) {44} edge (n4);
      \node (n10) at (-0.5,-3) {14} edge (n4);
      \node (n11) at (0.5,-3) {12} edge (n5);
      \node<2->[onslide=<2-6>{highlight base}]
          (n12) at (1.5,-3)
          {\alt<7->{10}{\alt<5->{9}{11}}}
        edge (n5);
      \node<11->[onslide=<11-12>{highlight base}]
          (n13) at (2.5,-3)
          {\alt<13->{60}{2}}
        edge (n6);

      \draw<3-4,7-9> (n12) edge[bend right=45,dotted]
                  node[draw=none,above,sloped]{$<$} (n5);
      \draw<6>   (n12) edge[bend right=45,dotted]
                  node[draw=none,above,color=red,sloped]{$>$} (n5);
      \draw<9>   (n5) edge[bend left=45,dotted]
                  node[draw=none,above,sloped]{$>$} (n2);
      \draw<12>  (n13) edge[bend left=45,dotted]
                  node[draw=none,above,color=red,sloped]{$<$} (n6);
      \draw<13-15> (n13) edge[bend left=45,dotted]
                  node[draw=none,above,sloped]{$>$} (n6);
      \draw<13>  (n6) edge[bend right=45,dotted]
                  node[draw=none,above,color=red,sloped]{$>$} (n2);
      \draw<14-15>  (n6) edge[bend right=45,dotted]
                  node[draw=none,above,sloped]{$<$} (n2);
      \draw<14>   (n2) edge[bend right=45,dotted]
                  node[draw=none,above,color=red,sloped]{$>$} (n0);
      \draw<15>  (n2) edge[bend right=45,dotted]
                  node[draw=none,above,sloped]{$<$} (n0);
      \draw<16>  (n1) edge[bend left=45,dotted]
                  node[draw=none,above,sloped]{?} (n0);
  \end{tikzpicture}
  \hfill\strut
\end{center}

  \bigskip
  \uncover<17->{
    How long does this take?
    \uncover<18->{How tall is the tree?}
    \LARGE
    \huncover{19}{\bigO{\log n}}%
  }
\end{frame}

\begin{frame}[fragile]{Operation \FN{RemoveMin}}
  \begin{overprint}
    \onslide<-7>
    \uncover<2->{Step 1: Replace the root with the \emph{last} node, and
      remove the last node. \uncover<3->{(This preserves tree
    completeness.)}}
    \par\smallskip
    \uncover<4->{Step 2: Restore the invariant by ``percolating down'':
      swap new node with its \emph{smaller} {child} until invariant
    is restored.}
    \onslide<8->
    Why do we swap with the smaller child? \uncover<9->{Transitivity again!}
    \par\medskip
    \uncover<10->{
      How long does this take?
      \uncover<11->{How tall is the tree?}
      \LARGE
      \huncover{12}{\bigO{\log n}}%
    }
  \end{overprint}
  \begin{center}
  \strut\hfill
  \begin{tikzpicture}
      [
        every node/.style={
          circle,
          draw=black,
          inner sep=2pt,
        },
      ]
      \node[onslide=<2-5>{highlight base}]
      (n0) at (0,0) {\alt<7->{8}{\alt<6->{3}{\alt<3->{60}{2}}}};
      \node (n1) at (-2,-1) {5} edge (n0);
      \node[onslide=<6>{highlight base}]
      (n2) at (2,-1) {\alt<7->{8}{\alt<6->{60}{3}}}
        edge (n0);
      \node (n3) at (-3,-2) {17} edge (n1);
      \node (n4) at (-1,-2) {6} edge (n1);
      \node (n5) at (1,-2) {9} edge (n2);
      \node[onslide=<7>{highlight base}]
      (n6) at (3,-2) {\alt<7->{60}{8}} edge (n2);
      \node (n7) at (-3.5,-3) {20} edge (n3);
      \node (n8) at (-2.5,-3) {37} edge (n3);
      \node (n9) at (-1.5,-3) {44} edge (n4);
      \node (n10) at (-0.5,-3) {14} edge (n4);
      \node (n11) at (0.5,-3) {12} edge (n5);
      \node (n12) at (1.5,-3)
            {10}
            edge (n5);
      \node<-2>[onslide=<2>{highlight base}] (n13) at (2.5,-3)
            {60}
            edge (n6);

      \draw<5>   (n0) edge[bend left=45,dotted]
                  node[draw=none,above,color=red,sloped]{$>$} (n2);
      \draw<6-8> (n0) edge[bend left=45,dotted]
                  node[draw=none,above,sloped]{$<$} (n2);
      \draw<6>   (n2) edge[bend left=45,dotted]
                  node[draw=none,above,color=red,sloped]{$>$} (n6);
      \draw<7-8> (n2) edge[bend left=45,dotted]
                  node[draw=none,above,sloped]{$<$} (n6);
  \end{tikzpicture}
  \hfill\strut
\end{center}

\end{frame}

\takeaways

\end{document}
