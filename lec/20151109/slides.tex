\RequirePackage[l2tabu, orthodox]{nag}

\documentclass{beamer}

\usepackage{microtype}
\usepackage{amsmath,amssymb}
%??? :( \usepackage[MnSymbol]{mathspec}
\usepackage{mathspec}
\usepackage[final]{listings}
\usepackage{graphicx}

\setallmainfonts{TeX Gyre Schola}
\setallmonofonts{Menlo}

%\usepackage{xunicode}
%\usepackage{xltxtra}
%\usepackage{graphicx}
%\usepackage{forloop}
%\usepackage{stmaryrd}
%\usepackage{mathtools}
%\usepackage{fancyvrb}

\usepackage{tikz}
% \usetikzlibrary{arrows.meta}
\usetikzlibrary{backgrounds}
\usetikzlibrary{calc}
% \usetikzlibrary{datavisualization}
% \usetikzlibrary{datavisualization.formats.functions}
\usetikzlibrary{fit}
% \usetikzlibrary{positioning}
% \usetikzlibrary{scopes}
\usetikzlibrary{shapes.geometric}

\def\bigO#1{\ensuremath{\mathcal O(#1)}}
\newcommand\fun[1]{\ensuremath{\lambda #1.\,}}
\renewcommand\T[1]{\ensuremath{T_{\mathit{#1}}}}

%%%
%%% BASIC SETTINGS
%%%

\frenchspacing
\unitlength=0.01\textwidth
\thicklines
\urlstyle{sf}
\graphicspath{{images/}}
\setlength{\parskip}{.25em}

\defaultfontfeatures{
    Mapping=tex-text,
    Scale=MatchLowercase,
}

%%%
%%% LISTINGS SETUP
%%%

\lstdefinelanguage{ASL}%
  {sensitive,%
   alsoletter=-+*/\#?,%
   alsodigit=-,%
   morekeywords={%
      and,%
      begin,%
      check-error,%
      check-expect,%
      check-within,%
      cond,%
      define,%
      define-struct,%
      else,%
      if,%
      lambda,%
      local,%
      or,%
   },%
   morekeywords=[2]{%
     cons,%
     \#true,%
     \#false,%
   },%
   morekeywords=[3]{%
     this-is-here-because-suck,%
     cons?,%
     empty?,%
     -,+,*,/,%
     first,%
     rest,%
   }%,
   morecomment=[l];,%
   morecomment=[s]{\#|}{|\#},%
   morestring=[b]"%
  }[keywords,comments,strings]%

\lstnewenvironment{asl}[1][]
{%
  \begingroup
  \lstset{language=ASL,#1}%
}
{%
  \endgroup
}

\lstnewenvironment{cpp}[1][]
{%
  \begingroup
  \lstset{language=C++,#1}%
}
{%
  \endgroup
}


\newcommand{\textcode}{%
  \lstset{language=ASL,basicstyle=\ttfamily\small}%
  \lstinline
}

\lstset{%
  numberstyle=\scriptsize,
  showspaces=false,
  basicstyle=\ttfamily,
  keywordstyle=\ttfamily\color{red!70!black},
  keywordstyle=[2]\ttfamily\color{blue!70!black},
  keywordstyle=[3]\ttfamily\color{green!40!black},
  numberstyle=\ttfamily\color{blue!70!black},
  commentstyle=\itshape\color{violet!80!black},
  escapeinside={($}{$)},
}

\newcommand{\textasl}{%
  \lstset{language=ASL,basicstyle=\ttfamily\small}%
  \lstinline
}

\let\sup^

\catcode`\^\active\relax
\def^#1^{\textasl|#1|}

%%%
%%% BEAMER SETUP
%%%

\setbeamercolor{frametitle}{fg=black}
\setbeamercolor{normal text}{fg=black}

\usefonttheme{serif}

\setbeamertemplate{frametitle}
  {\begin{center}\medskip
   \insertframetitle\par
   \end{center}}

\setbeamertemplate{itemize item}{$\bullet$}

\setbeamertemplate{navigation symbols}{}
\setbeamertemplate{footline}[text line]{%
    \hfill\strut{%
        \small\color{black!75}%
        \texttt{\insertframenumber:\insertframeslidenumber}%
    }%
    \hfill%
}

\makeatletter
\newcount\frameslidetemp
\newcommand\insertframeslidenumber{%
  \frameslidetemp\c@page\relax
  \advance\frameslidetemp-\beamer@startpageofframe\relax
  \advance\frameslidetemp1\relax
  \the\frameslidetemp
}
\makeatother

%%%
%%% GENERALLY USEFUL LITTLE MACROS
%%%

% For using \alt within verbatim:
\newcommand\ALT[3]{\alt<#1>{#2}{#3}}

% Or: \ALT = \S\alt
\def\S#1#2{#2<#1>}

\newcommand<>\always[1]{#1}

\newcommand\widen[1][1.2]{%
    \setlength\baselineskip{#1\baselineskip}%
}

%%%
%%% COLORS
%%%

\newcommand\K[1]{\textcolor{orange!70!black}{#1}}
\newcommand\REF{\textcolor{green!50!black}{ref}}

%%%
%%% TIKZ
%%%

% \place[STYLE]{LOCATION}{CONTENT} places CONTENT at LOCATION, applying
% STYLE
\newcommand\place[3][]{%
  \tikz[orp] \node[#1] at (#2) {#3};%
}

% \anchor{NAME} creates an addressable TikZ coordinate at the current
% location.
\newcommand\anchor[1]{%
  \tikz[orp] \coordinate (#1);%
}

% \nil{NODE} draws a slash across the node; if NODE is a square then we
% get the standard graphic notation for null/nil.
\newcommand\nil[1][1em]{%
  \begin{tikzpicture}[nil]
    \node[pointer tail, line width=0, opacity=0] {};
    \draw[overlay]
      node[pointer tail] (dummy) {}
      (dummy.west |- dummy.south) -- (dummy.east |- dummy.north);
  \end{tikzpicture}%
}

\makeatletter
  \newcommand\cons[1][]{\cons@kont{#1}}
  \def\cons@kont#1#2(#3)#4#5{\struct@kont{#1}{}(#3){#4&#5}}

  \newcommand\struct[1][]{\struct@kont{#1}}
  \def\struct@kont#1#2(#3)#4{
      \node[draw, inner sep=0, #1] (#3) {%
        \begin{tabular}{|*{100}{c|}}
          \hline
          #4\\
          \hline
        \end{tabular}
      }
  }
\makeatother

% \genpointer[STYLE]{BEND}{TARGET}
\newcommand\genpointer[2][genpointer]{%
  \tikz[remember picture, #1]
    \draw
      node[pointer tail] {}
      edge[->, overlay] (#2) ;%
}

% \hpointer[BEND]{TARGET} for a highlighted pointer
\newcommand\hpointer[2][]{%
  \genpointer[highlit pointer,#1]{#2}%
}

% \ppointer[BEND]{TARGET} for a non-highlighted pointer
\newcommand\ppointer[2][]{%
  \genpointer[pointer,#1]{#2}%
}

% \pointer[BEND]{TARGET} is highlighted if used in a highlighted context
\newcommand\pointer[2][]{%
  \ifhighlit\let\pointertemp\hpointer\else\let\pointertemp\ppointer\fi
  \pointertemp[#1]{#2}%
}

\newcommand\genpointerbase[2][genpointer]{%
  \tikz[remember picture, #1]
    \draw node[pointer tail] (#2) {};%
}

\newcommand\ppointerbase[2][genpointer]{%
  \genpointerbase[pointer,#1]{#2}%
}

\newcommand\hpointerbase[2][genpointer]{%
  \genpointerbase[highlit pointer,#1]{#2}%
}

\newcommand\pointerbase[2][]{%
  \ifhighlit\let\pointertemp\hpointerbase\else\let\pointertemp\ppointerbase\fi
  \pointertemp[#1]{#2}%
}

\newcommand\genpointerpoint[3][genpointer]{
  \draw (#2) edge[->, remember picture, overlay, #1] (#3) ;
}

\newcommand\ppointerpoint[2][genpointer]{%
  \genpointerpoint[pointer,#1]{#2}%
}

\newcommand\hpointerpoint[2][genpointer]{%
  \genpointerpoint[highlit pointer,#1]{#2}%
}

\newcommand\pointerpoint[2][]{%
  \ifhighlit\let\pointertemp\hpointerpoint\else\let\pointertemp\ppointerpoint\fi
  \pointertemp[#1]{#2}%
}

\tikzset{
  onslide/.code args={<#1>#2}{%
    \only<#1>{\pgfkeysalso{#2}}
  },
  orp/.style={
    overlay,
    remember picture,
  },
  rem/.style={
    remember picture,
  },
  every node/.style={
    very thick,
    inner sep=3pt,
  },
  treenode/.style={
    circle,
    draw=black,
    inner sep=2pt,
  },
  subtree/.style={
    treenode,
    isosceles triangle,
    isosceles triangle apex angle=30,
    text width=1em,
    align=center,
    shape border rotate=90,
    anchor=apex,
    inner xsep=0pt,
    inner ysep=4pt,
    fill=orange!30!white,
    draw=orange!50!black,
  },
  balance/.style={
    draw=none,
    color=green!70!black,
    anchor=north,
  },
  bad balance/.style={
    balance,
    color=red,
    },
  every path/.style={very thick},
  ref/.style={
    draw=red,
    densely dotted,
    font=\ttfamily,
  },
  cons/.style={
    draw=black,
    font=\ttfamily,
  },
  genpointer/.style={
    solid,
    bend left=30,
    >=Latex,
  },
  pointer/.style={
    genpointer,
    color=blue!50!black,
    opacity=.5,
  },
  highlit pointer/.style={
    genpointer,
    color=red!70!black,
  },
  nil/.style={
    color=black!50,
  },
  pointer tail/.style={
    draw,
    circle,
    text width=0,
    text height=1ex,
    inner sep=0,
  },
  highlight base/.style={
    outer color=#1!50,
    inner color=#1!30,
  },
  highlight base/.default=yellow,
  highlight/.style={
    highlight base,
    color=yellow!50!white!90!black,
    draw opacity=0.5,
    thick,
    draw,
    rounded corners=2pt,
  },
  bintree/.style={
    level 1/.style={sibling distance=#1},
    level 2/.style={sibling distance=#1/2},
    level 3/.style={sibling distance=#1/4},
    level 4/.style={sibling distance=#1/8},
    level 5/.style={sibling distance=#1/16},
  },
  bintree/.default={8em},
}

\newif\ifhighlit
\highlitfalse

% \highlight<WHEN>[NAME]{CONTENT}
\newcommand<>\highlight[2][\relax]{%
  \ifx#1\relax
    \begin{tikzpicture}
  \else
    \begin{tikzpicture}[remember picture]
  \fi
    \node[
      inner sep=0pt,
      outer sep=0pt,
      text depth=0pt,
      align=left,
    ] (highlight temp) {%
        \only#3{\highlittrue}%
        #2%
        \only#3{\highlitfalse}%
    };
    \begin{scope}[on background layer]
      \coordinate (highlight temp 1) at
        ($(highlight temp.north east) + (2pt,2pt)$);
      \coordinate (highlight temp 2) at
        ($(highlight temp.base west) + (-2pt,-3pt)$);
      \node#3 [
        overlay,
        highlight,
        inner sep=0pt,
        fit=(highlight temp 1) (highlight temp 2),
      ] {};
    \end{scope}
    \ifx#1\relax
      \relax
    \else
      \node (#1) [
        overlay,
        inner sep=0pt,
        fit=(highlight temp 1) (highlight temp 2),
      ] {};
    \fi
  \end{tikzpicture}%
}

\newcommand\HI[2]{\highlight<#1>{#2}}

% \huncover[NAME]{STEP}{CONTENT} uncovers from STEP onward, highlighting
% at STEP but not thereafter.
\newcommand\huncover[3][\relax]{%
  \uncover<#2->{\highlight<#2>[#1]{#3}}%
}

% \huncover[NAME]{STEP}{CONTENT} onlys from STEP onward, highlighting
% at STEP but not thereafter.
\newcommand\honly[3][\relax]{%
  \only<#2->{\highlight<#2>[#1]{#3}}%
}

\begin{document}

%%%
%%% TITLE PAGE
%%%

\newcommand{\takeaways}{
  \begin{frame}{Take-aways}
    \begin{itemize}
      \item What is the AVL property?
      \item How does AVL tree insertion maintain the property?
    \end{itemize}
  \end{frame}
}

\newcommand{\FN}[1]{\ensuremath{\mathop{\mathit{#1}}}}
\newcommand{\PrioQ}{\ensuremath{\mathsf{PrioQ}}}
\newcommand{\Element}{\ensuremath{\mathsf{Element}}}
\newcommand{\Bool}{\ensuremath{\mathsf{Bool}}}

\begin{frame}[fragile]\relax
  \thispagestyle{empty}

  %% Need to initialize Tikz opacity inside a frame, not outside:
  \begin{tikzpicture}[opacity=0,overlay]\end{tikzpicture}

  \begin{center}
    {\Huge AVL\footnote{Georgy \textbf Adelson-\textbf Velsky and
    Evgenii \textbf Landis} Trees}

  \bigskip
  \Large
  EECS 214

  \medskip
  November 9, 2015
  \end{center}
\end{frame}

\takeaways

\begin{frame}[fragile]{A basic binary tree}
  A binary tree, describing structure but not content:
  \begin{asl}
; An [BinTree X] is one of:
; -- (leaf)
; -- (branch [BinTree X] X [BinTree X])
(define-struct leaf [])
(define-struct branch [left element right])

; A [BST X] is a [BinTree X]
; that satisfies the BST property
  \end{asl}
\end{frame}

\begin{frame}[fragile]{Problem!}
  Binary search is \bigO{\log n}, right?
  \begin{center}
  \begin{tikzpicture}[
        every node/.style={
          circle,
          draw=black,
          inner sep=2pt,
        }
      ]
    \node (n1) at (0,0) {1};
    \uncover<2->{\node (n2) at (.5,-.5) {2} edge (n1);}
    \uncover<3->{\node (n3) at (1,-1) {3} edge (n2);}
    \uncover<4->{\node (n4) at (1.5,-1.5) {4} edge (n3);}
    \uncover<5->{\node (n5) at (2.0,-2.0) {5} edge (n4);}
    \uncover<6->{\node (n6) at (2.5,-2.5) {6} edge (n5);}
    \uncover<7->{\node (n7) at (3.0,-3.0) {7} edge (n6);}
    \uncover<8->{\node (n8) at (3.5,-3.5) {8} edge (n7);}
  \end{tikzpicture}
  \end{center}
  \uncover<9->{Only if the tree is (sufficiently) balanced.}
\end{frame}

\begin{frame}{Solution}
  We need some balance.

  \pause
  \bigskip
  There is a variety of self-balancing trees\ldots
\end{frame}

\newcommand{\invars}[3]{
    %\begin{minipage}{2in}
      \raggedright
      \textbf{Main idea:}~#1
      \par\bigskip
      \textbf{Local invariant:}~#2
      \par\bigskip
      \textbf{Global invariant:}~#3
  %\end{minipage}
}


\begin{frame}[fragile]{Red-black trees}
  \invars
    {Every node has an extra bit marking it “red” or “black”}
    {No red node has red children}
    {Equal number of black nodes from root to every leaf}
\end{frame}

\begin{frame}[fragile]{2-3 trees}
  \invars
    {2-nodes have one element and two children; 3-nodes have two
    elements and three children}
    {All subtrees of a node have the same height}
    {Every leaf is at the same depth}
\end{frame}

\begin{frame}[fragile]{2-4 trees}
  \invars
    {Like 2-3 trees, but also has 4-nodes with three elements and four
    children.}
    {All subtrees of a node have the same height}
    {Every leaf is at the same depth}
\end{frame}

\begin{frame}[fragile]{B-trees}
  \invars
    {Generalization of 2-4 trees to 2-$k$ trees}
    {Like 2-4 trees, but allow some number of missing subtrees}
    {Every leaf is at the same depth}
\end{frame}

\begin{frame}[fragile]{Splay trees}
  \invars
    {Cache recently access elements near the root of the tree}
    {\emph{Need amortized analysis to talk about this}}
    {Paths are \emph{very likely} to be \bigO{\log n}}
\end{frame}

\begin{frame}[fragile]{AVL trees}
  \invars
    {Maintain a \emph{balance factor} giving the difference between
     each node’s subtrees’ heights}
    {Balance factor is at most 1}
    {Tree is approximately height-balanced}
\end{frame}

\begin{frame}{Big theme!}
  \begin{center}
    \LARGE
    We can ensure a global property by maintaining a local property
  \end{center}
\end{frame}

\begin{frame}[fragile]{AVL tree data definition}
Each branch includes a balance factor of type ^B^:
\begin{asl}
(define-struct leaf [])
(define-struct branch [balance left element right])

; A [PreAVLTree B X] is one of:
; -- (make-leaf)
; -- (make-branch B [PreAVLTree B X]
;                 X [PreAVLTree B X])
; satisfying the BST property

; An [AVLTree X] is [PreAVLTree [-1, 1] X]
; satisfying the AVL property as well
\end{asl}
\end{frame}

\begin{frame}[fragile]{Defining the AVL property}
  The AVL property relies on balance factors, so it requires that
  balance factors be accurate.
  \\\hfill\emph{See function ^accurate-balances?^ in ^avl.rkt^.}

  \par\bigskip
  Then we require that every balance factor be ^-1^, ^0^, or ^1^.
  \\\hfill\emph{See function ^avl-balances?^ in ^avl.rkt^.}

  \par\bigskip
\begin{asl}
; avl? : [PreAVLTree Integer X] -> Boolean
; Is the tree actually an AVL tree?
(define (avl? tree)
  (and (bst? tree)
       (accurate-balances? tree)
       (avl-balances? tree)))
\end{asl}
\end{frame}

\begin{frame}[fragile]{Maintaining the AVL property}
  Suppose we have an AVL tree:

  \begin{center}
  \begin{tikzpicture}[every node/.style={treenode}]
    \draw
    node (B) at (0, 0) {B}
      node[balance] at (B.south) {0}
    node[subtree] (A) at (-.5,-.5) {A}
    node[subtree] (C) at (.5, -.5) {C}
    (A.apex) -- (B)
    (C.apex) -- (B)
    ;
  \end{tikzpicture}\\
\small(Convention: triangles represent equal-height subtrees.)
\end{center}
\pause
Right now the balance factor is ^0^. So if we insert into A or C and
that subtree grows in height, it becomes ^-1^ or ^1^.
\end{frame}

\begin{frame}[fragile]{Maintaining the AVL property}
  \begin{center}
  \begin{tikzpicture}[every node/.style={treenode}]
    \draw
    node (B) at (0, 0) {B}
      node[balance] at (B.south) {1}
    node[subtree] (A) at (-.5,-.5) {A}
      (A.apex) -- (B)
    node (D) at (1,-1) {D} edge (B)
      node[balance] at (D.south) {0}
    node[subtree] (C) at (.5, -1.5) {C}
      (C.apex) -- (D)
    node[subtree] (E) at (1.5, -1.5) {E}
      (E.apex) -- (D)
    ;
  \end{tikzpicture}\\
\end{center}
Right now the balance factor at B is ^1^.
\par
\begin{overprint}
  \onslide<1-2>
Suppose we insert into A. What happens to B's balance factor?
  \begin{itemize}
    \item<2-> If no change in A's height, then B's balance doesn't change
    \item<2-> If A's height increases, then B's balance is now ^0^
  \end{itemize}
  \onslide<3-5>
   Suppose we insert into C. What happens to B's balance factor?
  \begin{itemize}
    \item<4-> If no change, then B's balance doesn't change
    \item<4-> If increase, then B's balance becomes
      \alert{^2^}\uncover<5->{---not okay!}
  \end{itemize}
  \onslide<6>
   Likewise, suppose we insert into E. What happens to B's balance factor?
  \begin{itemize}
    \item If no change, then B's balance doesn't change
    \item If increase, then B's balance becomes
      \alert{^2^}---not okay!
  \end{itemize}
\end{overprint}
\end{frame}

\begin{frame}[fragile]{Right-right case}
    If the height the right-right subtree (formerly E) increases, we
    get a situation like this:
  \begin{center}
  \begin{tikzpicture}[
      every node/.style={treenode},
      baseline=-2,
  ]
    \draw
    node (B) at (0, 0) {B}
      node[bad balance] at (B.south) {2}
    node[subtree] (A) at (-.5,-.5) {A}
      (A.apex) -- (B)
    node (D) at (1,-1.5) {D} edge (B)
      node[balance] at (D.south) {1}
    node[subtree] (C) at (.5, -2) {C}
      (C.apex) -- (D)
    node (F) at (2,-3) {F} edge (D)
      node[balance] at (F.south) {0}
    node[subtree] (E) at (1.5, -3.5) {E}
      (E.apex) -- (F)
    node[subtree] (G) at (2.5, -3.5) {G}
      (G.apex) -- (F)
    ;
  \end{tikzpicture}%
  \uncover<2->{$\quad\Longrightarrow\quad$%
  \begin{tikzpicture}[
      every node/.style={treenode},
      baseline=-2,
  ]
    \draw
    node (D) at (0, 0) {D}
      node[balance] at (D.south) {0}
    node (B) at (-1,-1) {B} edge (D)
      node[balance] at (B.south) {0}
    node (F) at (1,-1) {F} edge (D)
      node[balance] at (F.south) {0}
    node[subtree] (A) at (-1.5,-1.5) {A}
      (A.apex) -- (B)
    node[subtree] (C) at (-.5, -1.5) {C}
      (C.apex) -- (B)
    node[subtree] (E) at (.5, -1.5) {E}
      (E.apex) -- (F)
    node[subtree] (G) at (1.5, -1.5) {G}
      (G.apex) -- (F)
    ;
  \end{tikzpicture}%
  }
\end{center}
\end{frame}

\begin{frame}[fragile]{Right-left case}
    If the height the right-right subtree (formerly C) increases, we
    get a situation like this:
  \begin{center}
  \begin{tikzpicture}[
      every node/.style={treenode},
      baseline=-2,
  ]
    \draw
    node (B) at (0, 0) {B}
      node[bad balance] at (B.south) {2}
    node[subtree] (A) at (-.5,-.5) {A}
      (A.apex) -- (B)
    node (F) at (2,-1.5) {F} edge (B)
      node[balance] at (F.south) {-1}
    node[subtree] (G) at (2.5, -2) {G}
      (G.apex) -- (F)
    node (D) at (1,-3) {D} edge (F)
      node[balance] at (D.south) {0}
    node[subtree] (C) at (0.5, -3.5) {C}
      (C.apex) -- (D)
    node[subtree] (E) at (1.5, -3.5) {E}
      (E.apex) -- (D)
    ;
  \end{tikzpicture}%
  \only<2->{$\quad\Longrightarrow\quad$%
  \begin{tikzpicture}[
      every node/.style={treenode},
      baseline=-2,
  ]
    \draw
    node (B) at (0, 0) {B}
      node[bad balance] at (B.south) {2}
    node[subtree] (A) at (-.5,-.5) {A}
      (A.apex) -- (B)
    node (D) at (1,-1.5) {D} edge (B)
      node[balance] at (D.south) {1}
    node[subtree] (C) at (.5, -2) {C}
      (C.apex) -- (D)
    node (F) at (2,-3) {F} edge (D)
      node[balance] at (F.south) {0}
    node[subtree] (E) at (1.5, -3.5) {E}
      (E.apex) -- (F)
    node[subtree] (G) at (2.5, -3.5) {G}
      (G.apex) -- (F)
    ;
  \end{tikzpicture}%
  }
  \end{center}
  \uncover<3->{This is just the right-right case, which we know how to
  handle.}
\end{frame}

\takeaways

\end{document}
