\RequirePackage[l2tabu, orthodox]{nag}

\documentclass{beamer}

\usepackage{microtype}
\usepackage{amsmath,amssymb}
%??? :( \usepackage[MnSymbol]{mathspec}
\usepackage{mathspec}
\usepackage[final]{listings}
\usepackage{graphicx}

\setallmainfonts{TeX Gyre Schola}
\setallmonofonts{Menlo}

%\usepackage{xunicode}
%\usepackage{xltxtra}
%\usepackage{graphicx}
%\usepackage{forloop}
%\usepackage{stmaryrd}
%\usepackage{mathtools}
%\usepackage{fancyvrb}

\usepackage{tikz}
% \usetikzlibrary{arrows.meta}
% \usetikzlibrary{backgrounds}
\usetikzlibrary{calc}
% \usetikzlibrary{datavisualization}
% \usetikzlibrary{datavisualization.formats.functions}
\usetikzlibrary{math}
% \usetikzlibrary{fit}
% \usetikzlibrary{positioning}
% \usetikzlibrary{scopes}
% \usetikzlibrary{shapes.geometric}

\def\bigO#1{\ensuremath{\mathcal O(#1)}}
\def\BigO#1{\ensuremath{\mathcal O\left(#1\right)}}
\newcommand\fun[1]{\ensuremath{\lambda #1.\,}}
\renewcommand\T[1]{\ensuremath{T_{\mathit{#1}}}}

%%%
%%% BASIC SETTINGS
%%%

\frenchspacing
\unitlength=0.01\textwidth
\thicklines
\urlstyle{sf}
\graphicspath{{images/}}
\setlength{\parskip}{.25em}

\defaultfontfeatures{
    Mapping=tex-text,
    Scale=MatchLowercase,
}

%%%
%%% LISTINGS SETUP
%%%

\lstdefinelanguage{ASL}%
  {sensitive,%
   alsoletter=-+*/\#?,%
   alsodigit=-,%
   morekeywords={%
      and,%
      begin,%
      check-error,%
      check-expect,%
      check-within,%
      cond,%
      define,%
      define-struct,%
      else,%
      if,%
      lambda,%
      local,%
      or,%
   },%
   morekeywords=[2]{%
     cons,%
     \#true,%
     \#false,%
   },%
   morekeywords=[3]{%
     this-is-here-because-suck,%
     cons?,%
     empty?,%
     -,+,*,/,%
     first,%
     rest,%
   }%,
   morecomment=[l];,%
   morecomment=[s]{\#|}{|\#},%
   morestring=[b]"%
  }[keywords,comments,strings]%

\lstnewenvironment{ASL}[1][]
{%
  \begingroup
  \lstset{language=ASL,#1}%
}
{%
  \endgroup
}

\lstnewenvironment{CXX}[1][]
{%
  \begingroup
  \lstset{language=C++,#1}%
}
{%
  \endgroup
}

\lstnewenvironment{Java}[1][]
{%
  \begingroup
  \lstset{language=Java,#1}%
}
{%
  \endgroup
}

\lstnewenvironment{Python}[1][]
{%
  \begingroup
  \lstset{language=Python,#1}%
}
{%
  \endgroup
}


\lstnewenvironment{Ruby}[1][]
{%
  \begingroup
  \lstset{language=Ruby,#1}%
}
{%
  \endgroup
}

\newcommand{\textcode}{%
  \lstset{language=ASL,basicstyle=\ttfamily\small}%
  \lstinline
}

\lstset{%
  numberstyle=\scriptsize,
  showspaces=false,
  basicstyle=\ttfamily,
  keywordstyle=\ttfamily\color{red!70!black},
  keywordstyle=[2]\ttfamily\color{blue!70!black},
  keywordstyle=[3]\ttfamily\color{green!40!black},
  numberstyle=\ttfamily\color{blue!70!black},
  commentstyle=\itshape\color{violet!80!black},
  escapeinside={($}{$)},
}

\newcommand{\textasl}{%
  \lstset{language=ASL,basicstyle=\ttfamily\small}%
  \lstinline
}

\let\sup^

\catcode`\^\active\relax
\def^#1^{\textasl|#1|}

%%%
%%% BEAMER SETUP
%%%

\setbeamercolor{frametitle}{fg=black}
\setbeamercolor{normal text}{fg=black}

\usefonttheme{serif}

\setbeamertemplate{frametitle}
  {\begin{center}\medskip
   \insertframetitle\par
   \end{center}}

\setbeamertemplate{itemize item}{$\bullet$}

\setbeamertemplate{navigation symbols}{}
\setbeamertemplate{footline}[text line]{%
    \hfill\strut{%
        \small\color{black!75}%
        \texttt{\insertframenumber:\insertframeslidenumber}%
    }%
    \hfill%
}

\makeatletter
\newcount\frameslidetemp
\newcommand\insertframeslidenumber{%
  \frameslidetemp\c@page\relax
  \advance\frameslidetemp-\beamer@startpageofframe\relax
  \advance\frameslidetemp1\relax
  \the\frameslidetemp
}
\makeatother

%%%
%%% GENERALLY USEFUL LITTLE MACROS
%%%

% For using \alt within verbatim:
\newcommand\ALT[3]{\alt<#1>{#2}{#3}}

% Or: \ALT = \S\alt
\def\S#1#2{#2<#1>}

\newcommand<>\always[1]{#1}

\newcommand\widen[1][1.2]{%
    \setlength\baselineskip{#1\baselineskip}%
}

%%%
%%% COLORS
%%%

\newcommand\K[1]{\textcolor{orange!70!black}{#1}}
\newcommand\REF{\textcolor{green!50!black}{ref}}

%%%
%%% TIKZ
%%%

% \place[STYLE]{LOCATION}{CONTENT} places CONTENT at LOCATION, applying
% STYLE
\newcommand\place[3][]{%
  \tikz[orp] \node[#1] at (#2) {#3};%
}

% \anchor{NAME} creates an addressable TikZ coordinate at the current
% location.
\newcommand\anchor[1]{%
  \tikz[orp] \coordinate (#1);%
}

% \nil{NODE} draws a slash across the node; if NODE is a square then we
% get the standard graphic notation for null/nil.
\newcommand\nil[1][1em]{%
  \begin{tikzpicture}[nil]
    \node[pointer tail, line width=0, opacity=0] {};
    \draw[overlay]
      node[pointer tail] (dummy) {}
      (dummy.west |- dummy.south) -- (dummy.east |- dummy.north);
  \end{tikzpicture}%
}

\makeatletter
  \newcommand\cons[1][]{\cons@kont{#1}}
  \def\cons@kont#1#2(#3)#4#5{\struct@kont{#1}{}(#3){#4&#5}}

  \newcommand\struct[1][]{\struct@kont{#1}}
  \def\struct@kont#1#2(#3)#4{
      \node[draw, inner sep=0, #1] (#3) {%
        \begin{tabular}{|*{100}{c|}}
          \hline
          #4\\
          \hline
        \end{tabular}
      }
  }
\makeatother

% \genpointer[STYLE]{BEND}{TARGET}
\newcommand\genpointer[2][genpointer]{%
  \tikz[remember picture, #1]
    \draw
      node[pointer tail] {}
      edge[->, overlay] (#2) ;%
}

% \hpointer[BEND]{TARGET} for a highlighted pointer
\newcommand\hpointer[2][]{%
  \genpointer[highlit pointer,#1]{#2}%
}

% \ppointer[BEND]{TARGET} for a non-highlighted pointer
\newcommand\ppointer[2][]{%
  \genpointer[pointer,#1]{#2}%
}

% \pointer[BEND]{TARGET} is highlighted if used in a highlighted context
\newcommand\pointer[2][]{%
  \ifhighlit\let\pointertemp\hpointer\else\let\pointertemp\ppointer\fi
  \pointertemp[#1]{#2}%
}

\newcommand\genpointerbase[2][genpointer]{%
  \tikz[remember picture, #1]
    \draw node[pointer tail] (#2) {};%
}

\newcommand\ppointerbase[2][genpointer]{%
  \genpointerbase[pointer,#1]{#2}%
}

\newcommand\hpointerbase[2][genpointer]{%
  \genpointerbase[highlit pointer,#1]{#2}%
}

\newcommand\pointerbase[2][]{%
  \ifhighlit\let\pointertemp\hpointerbase\else\let\pointertemp\ppointerbase\fi
  \pointertemp[#1]{#2}%
}

\newcommand\genpointerpoint[3][genpointer]{
  \draw (#2) edge[->, remember picture, overlay, #1] (#3) ;
}

\newcommand\ppointerpoint[2][genpointer]{%
  \genpointerpoint[pointer,#1]{#2}%
}

\newcommand\hpointerpoint[2][genpointer]{%
  \genpointerpoint[highlit pointer,#1]{#2}%
}

\newcommand\pointerpoint[2][]{%
  \ifhighlit\let\pointertemp\hpointerpoint\else\let\pointertemp\ppointerpoint\fi
  \pointertemp[#1]{#2}%
}

\tikzset{
  onslide/.code args={<#1>#2}{%
    \only<#1>{\pgfkeysalso{#2}}
  },
  orp/.style={
    overlay,
    remember picture,
  },
  rem/.style={
    remember picture,
  },
  every node/.style={
    very thick,
    inner sep=3pt,
  },
  treenode/.style={
    circle,
    draw=black,
    inner sep=2pt,
  },
  subtree/.style={
    treenode,
    isosceles triangle,
    isosceles triangle apex angle=30,
    text width=1em,
    align=center,
    shape border rotate=90,
    anchor=apex,
    inner xsep=0pt,
    inner ysep=4pt,
    fill=orange!30!white,
    draw=orange!50!black,
  },
  balance/.style={
    draw=none,
    color=green!70!black,
    anchor=north,
  },
  bad balance/.style={
    balance,
    color=red,
    },
  every path/.style={very thick},
  ref/.style={
    draw=red,
    densely dotted,
    font=\ttfamily,
  },
  cons/.style={
    draw=black,
    font=\ttfamily,
  },
  genpointer/.style={
    solid,
    bend left=30,
    >=Latex,
  },
  pointer/.style={
    genpointer,
    color=blue!50!black,
    opacity=.5,
  },
  highlit pointer/.style={
    genpointer,
    color=red!70!black,
  },
  nil/.style={
    color=black!50,
  },
  pointer tail/.style={
    draw,
    circle,
    text width=0,
    text height=1ex,
    inner sep=0,
  },
  highlight base/.style={
    outer color=#1!50,
    inner color=#1!30,
  },
  highlight base/.default=yellow,
  highlight/.style={
    highlight base,
    color=yellow!50!white!90!black,
    draw opacity=0.5,
    thick,
    draw,
    rounded corners=2pt,
  },
  bintree/.style={
    level 1/.style={sibling distance=#1},
    level 2/.style={sibling distance=#1/2},
    level 3/.style={sibling distance=#1/4},
    level 4/.style={sibling distance=#1/8},
    level 5/.style={sibling distance=#1/16},
  },
  bintree/.default={8em},
}

\newif\ifhighlit
\highlitfalse

% \highlight<WHEN>[NAME]{CONTENT}
\newcommand<>\highlight[2][\relax]{%
  \ifx#1\relax
    \begin{tikzpicture}
  \else
    \begin{tikzpicture}[remember picture]
  \fi
    \node[
      inner sep=0pt,
      outer sep=0pt,
      text depth=0pt,
      align=left,
    ] (highlight temp) {%
        \only#3{\highlittrue}%
        #2%
        \only#3{\highlitfalse}%
    };
    \begin{scope}[on background layer]
      \coordinate (highlight temp 1) at
        ($(highlight temp.north east) + (2pt,2pt)$);
      \coordinate (highlight temp 2) at
        ($(highlight temp.base west) + (-2pt,-3pt)$);
      \node#3 [
        overlay,
        highlight,
        inner sep=0pt,
        fit=(highlight temp 1) (highlight temp 2),
      ] {};
    \end{scope}
    \ifx#1\relax
      \relax
    \else
      \node (#1) [
        overlay,
        inner sep=0pt,
        fit=(highlight temp 1) (highlight temp 2),
      ] {};
    \fi
  \end{tikzpicture}%
}

\newcommand\HI[2]{\highlight<#1>{#2}}

% \huncover[NAME]{STEP}{CONTENT} uncovers from STEP onward, highlighting
% at STEP but not thereafter.
\newcommand\huncover[3][\relax]{%
  \uncover<#2->{\highlight<#2>[#1]{#3}}%
}

% \huncover[NAME]{STEP}{CONTENT} onlys from STEP onward, highlighting
% at STEP but not thereafter.
\newcommand\honly[3][\relax]{%
  \only<#2->{\highlight<#2>[#1]{#3}}%
}

\begin{document}

%%%
%%% TITLE PAGE
%%%

\newcommand{\takeaways}{
  \begin{frame}{Take-aways}
    \begin{itemize}
      \item What is \emph{amortized time}?
      \item How does amortized time differ from \emph{average time}?
      \item When is amortized time useful, and when might we want to
        avoid it?
      \item How can we figure out the amortized time of data structure
        operations?
      \item How does a dynamic array achieve its amortized time complexity?
    \end{itemize}
  \end{frame}
}

\newcommand{\FN}[1]{\ensuremath{\mathop{\mathit{#1}}}}
\newcommand{\PrioQ}{\ensuremath{\mathsf{PrioQ}}}
\newcommand{\Element}{\ensuremath{\mathsf{Element}}}
\newcommand{\Bool}{\ensuremath{\mathsf{Bool}}}

\begin{frame}[fragile]\relax
  \thispagestyle{empty}

  %% Need to initialize Tikz opacity inside a frame, not outside:
  \begin{tikzpicture}[opacity=0,overlay]\end{tikzpicture}

  \begin{center}
    {\Huge Amortized Analysis}

  \bigskip
  \Large
  EECS 214

  \medskip
  November 11--13, 2015
  \end{center}
\end{frame}

\takeaways

\begin{frame}{Example: dynamic arrays}
  \centering
  \begin{tabular}{l|l}
    \textbf{Language} & \textbf{Type} \\
    \hline
    C++ & \texttt{std::vector} \\
    Java & \texttt{ArrayList} \\
    Python & \texttt{list} \\
    Ruby & \texttt{Array} \\ \pause
    C & \emph{you’re on your own} \\ \pause
    ASL & \emph{you’re on your own} \\
  \end{tabular}
\end{frame}

\begin{frame}[fragile]{Iteratively growing a dynamic array}
  \begin{CXX}
std::vector<int> v;
for (int i = 0; i < N; ++i) v.push_back(i);
  \end{CXX}
  \begin{Java}
ArrayList<Integer> v = new ArrayList<>();
for (int i = 0; i < N; ++i) v.add(i);
  \end{Java}
  \begin{Python}
v = list()
for i in range(0, 10): v.append(i)
  \end{Python}
  \begin{Ruby}
v = Array.new
for i in 0 ... N do v.push(i) end
  \end{Ruby}
\end{frame}

\newdimen\xval
\newdimen\yval

\begin{frame}[fragile]{Time per operation}
  \begin{center}
  \xval=0pt
  \begin{tikzpicture}
    \foreach \yval in {
        1,
        2,
        3,1,
        5,1,1,1,
        9,1,1,1,1,1,1,1,
        17,1,1,1,1,1,1,1,1,1,1,1,1,1,1,1,
        33,1,1,1,1,1,1,1,1,1,1,1,1,1,1,1,1,1,1,1,1,1,1,1,1,1,1,1,1,1,1,1,
        65,1,1,1,1,1,1,1,1,1,1,1,1,1,1,1,1,1,1,1,1,1,1,1,1,1,1,1,1,1,1,1,
          1,1,1,1,1,1,1,1,1,1,1,1,1,1,1,1,1,1,1,1,1,1,1,1,1,1,1,1,1,1,1,1
      }
      {
        \tikzmath{
          \yval = 2 * \yval;
        }
        \draw[line width=4pt,blue]
          (\xval,0) edge (\xval,\yval pt);
        \global\advance\xval by 2pt
      }
  \end{tikzpicture}\\
  \end{center}
\end{frame}

\begin{frame}[fragile]{Accumulated time}
  \begin{center}
  \xval=0pt
  \yval=0pt
  \begin{tikzpicture}[yscale=.5]
    \foreach \dyval in {
        1,
        2,
        3,1,
        5,1,1,1,
        9,1,1,1,1,1,1,1,
        17,1,1,1,1,1,1,1,1,1,1,1,1,1,1,1,
        33,1,1,1,1,1,1,1,1,1,1,1,1,1,1,1,1,1,1,1,1,1,1,1,1,1,1,1,1,1,1,1,
        65,1,1,1,1,1,1,1,1,1,1,1,1,1,1,1,1,1,1,1,1,1,1,1,1,1,1,1,1,1,1,1,
          1,1,1,1,1,1,1,1,1,1,1,1,1,1,1,1,1,1,1,1,1,1,1,1,1,1,1,1,1,1,1,1
      }
      {
        \global\advance\yval by \dyval pt
        \draw[line width=4pt,blue]
        (\xval,0) edge (\xval,\yval);
        \global\advance\xval by 2pt
      }
  \end{tikzpicture}\\
  \end{center}
\end{frame}

\begin{frame}{What's it doing?}
  \begin{itemize}
    \item A dynamic array is backed by a fixed-size array with excess
      capacity: ^(define-struct dynarray [data size])^
    \item When the array fills, allocate a fixed-size array that's twice
      as big and copy over the elements.
  \end{itemize}
\end{frame}

\begin{frame}{Time complexity of a single insertion}
  \begin{align*}
    \intertext{A single insertion:}
    T_{\mathrm{insert}}(n) &= \bigO{n} \\
  \end{align*}
\end{frame}

\begin{frame}{Time complexity of a sequence of insertions}
  \begin{align*}
    \intertext{Hence, for a sequence of insertions:}
    T_{\mathrm{insert-sequence}}(m) &= \sum_{i=1}\sup{m} \bigO{i} \\
                                    &\uncover<2->{=
  \BigO{\sum_{i=1}\sup{m} i}} \\
                                    &\uncover<3->{= \bigO{1 + 2 + \cdots
+ (m - 1) + m}} \\
                                    &\uncover<4->{= \BigO{\frac{m(m +
1)}{2}}} \\
                                    &\uncover<5->{= \bigO{m\sup2}}
  \end{align*}
\end{frame}

\begin{frame}{Amortized time complexity}
  \centering
  Amortized time complexity considers the cost of a sequence of
  operations by paying attention to the state of the data structure.

  \pause\medskip
  Then it apportions the time evenly among the operations.

  \pause\medskip
  Amortization is about the \emph{worst case}, not merely the
  \emph{average case}.
\end{frame}

\begin{frame}{Banker’s method: real costs vs. accounting costs}
  Let $c_i$ be the actual cost of the $i$th operation

  Let $c'_i$ be the charged cost of the $i$th operation\pause---we choose
  this!

  \medskip
  \pause
  If total actual cost does not exceed the total charged cost,
  \[
    \sum_{i=1}\sup n c_i \le \sum_{i=1}\sup n c'_i\,,
  \]
  then we say that the $i$th operation has worst-case \emph{amortized}
  time $\bigO{c'_i}$,
\end{frame}

\begin{frame}<1-10>{Amortized time for dynamic array insertion (banker style)}
  Consider the $i$th insert operation (which results in size $i$):
    \[
      \begin{array}{c|rrrrrrrrrr}
        i & \phantom{-}1 & \phantom{-}2 & \phantom{-}3 & \phantom{-}4
        & \phantom{-}5 & \phantom{-}6 & \phantom{-}7 & \phantom{-}8
        & \phantom{-}9 & 10
        \uncover<2->{
        \\\hline
        \mathit{cap}_i & 1 & 2 & 4 & 4 & 8 & 8 & 8 & 8 & 16 & 16
        }
        \uncover<3->{
        \\
        \alt<-7>{
          \mathit{c}_i & 1 & 2 & 3 & 1 & 5 & 1 & 1 & 1 & 9 & 1
        }{
          \mathit{c}_i & 2 & 4 & 7 & 1 & 13 & 1 & 1 & 1 & 25 & 1
        }
        }
        \uncover<4->{
        \\
        \alt<-5>{
        \mathit{c'}_i & 1 & 1 & 1 & 1 & 1 & 1 & 1 & 1 & 1 & 1
        }{
          \alt<-6>{
            \mathit{c'}_i & 2 & 2 & 2 & 2 & 2 & 2 & 2 & 2 & 2 & 2
          }{
            \alt<-8>{
              \mathit{c'}_i & 3 & 3 & 3 & 3 & 3 & 3 & 3 & 3 & 3 & 3
            }{
              \alt<-9>{
                \mathit{c'}_i & 5 & 5 & 5 & 5 & 5 & 5 & 5 & 5 & 5 & 5
              }{
                \mathit{c'}_i & 7 & 7 & 7 & 7 & 7 & 7 & 7 & 7 & 7 & 7
              }
            }
          }
        }
        }
        \uncover<5->{
        \\
        \alt<-5>{
          \mathit{bal}_i & 0 & \alert{-1} & \alert{-1} & \alert{-1} &
          \alert{-1} & \alert{-1} & \alert{-1} &
          \llap{$\ldots$}
        }{
          \alt<-6>{
            \mathit{bal}_i & 1 & 1 & 0 & 1 & \alert{-1} & \alert{-1} & 0
            & 1 & \alert{-1} & \alert{-1}
          }{
            \alt<-7>{
              \mathit{bal}_i & 2 & 3 & 3 & 5 & 3 & 5 & 7 & 9 & 3 & 5
            }{
              \alt<-8>{
                \mathit{bal}_i & 1 & 0 & \alert{-1} & \alert{-1} & \llap{$\ldots$}
              }{
                \alt<-9>{
                  \mathit{bal}_i & 3 & 4 & 2 & 6 & \alert{-2} & \llap{$\ldots$}
                }{
                  \mathit{bal}_i & 5 & 8 & 8 & 14 & 8 & 14 & 20 & 26 & 8
                  & 14
                }
              }
            }
          }
        }
        }
      \end{array}
    \]

  \uncover<2->{
    Let $\mathit{cap}_i$ be the capacity after operation $i$
  }

  \uncover<3->{
    Let $c_i$ be the actual cost of the $i$th operation (number of
    elements inserted or copied)
  }

  \uncover<4->{
    Let $c'_i$ be the charged cost of the $i$th operation---we choose a
    constant to cover the cost of large future operations
  }

  \uncover<5->{
    Let $\mathit{bal}_i$ be the balance: $\mathit{bal}_i =
    \mathit{bal}_{i - 1} - c_i + c'_i$.
  }
\end{frame}

\begin{frame}{Physicist's method: potential ``energy''}
  We define a potential function $\Phi$ on data structure states, where:
  \begin{align*}
    \Phi(v_0) &= 0  & \text{starts at 0} \\
    \Phi(v_t) &\ge 0 & \text{never goes negative}
  \end{align*}
  \pause
  $\Phi$ is akin to the balance in the banker's method, but history-less

  \pause\bigskip
  We then define the amortized time of an operation:
  \begin{align*}
    c'_i &= c_i + \Phi(v_i) - \Phi(v_{i - 1}) \\\pause
         &= c_i + \Delta\Phi(v_i)
  \end{align*}
\end{frame}

\begin{frame}{Potential function for dynamic arrays}
  We choose a potential function
  \[
    \Phi(v) = 2n - m\,,
  \]
  where $n$ is the size and $m$ the capacity of $v$.

  \pause\medskip
  Let's check $\Phi$'s properties:
  \begin{itemize}
      \pause
    \item[$\checkmark$] The initial vector has no size and no capacity, so
      $\Phi(v_0) = 0$
      \pause
    \item[\uncover<6>{$\checkmark$}] The capacity is never more than twice the
      size, because we double when it's full\pause; hence $2n \ge
      m$\pause; hence
      $\Phi(v) = 2n - m \ge 0$.
  \end{itemize}
  %// We want to show 
\end{frame}

\begin{frame}{Amortized time for dynamic array insertion (physicist style)}
  Let's compute $c'_i$ for insertion. Remember that
  $c'_i = c_i + \Phi(v_i) - \Phi(v_{i - 1})$.
  There are two possibilities:
  \begin{itemize}
    \item<2->[\uncover<4->{$\checkmark$}]
      If $n < m$ then $c_i = 1$. \uncover<3->{So} \begin{align*}
        \uncover<3->{c'_i &= 1 + (2(n + 1) - m) - (2n - m)} \\
        \uncover<4->{&= 1 + 2 = 3}
      \end{align*}
    \item<5->[\uncover<8->{$\checkmark$}]
      If $n = m$ then $c_i = n + 1$ (copy plus simple insert).
      \uncover<6->{So} \begin{align*}
        \uncover<6->{c'_i &= n + 1 + (2(n + 1) - 2m) - (2n - m) \\}
        \uncover<7->{&= n + 1 + (2(n + 1) - 2n) - (2n - n)
            & \text{because $n = m$}\\}
        \uncover<8->{
               &= 1 + 2 + n + 2n - 2n + 2n - n = 3
        }
      \end{align*}
  \end{itemize}
\end{frame}

\begin{frame}{Another example: (na\"ive) persistent banker's queue}
  A data structure is \emph{persistent} when modifications do not destroy
  the previous state of the structure.\pause (The opposite is
  \emph{ephemeral}.)

  \medskip \pause
  What if we want a persistent FIFO queue with sub-linear operations?
\end{frame}

\takeaways

\end{document}
