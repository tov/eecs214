\RequirePackage[l2tabu, orthodox]{nag}

\documentclass{beamer}

\usepackage{microtype}
\usepackage{amsmath,amssymb}
%??? :( \usepackage[MnSymbol]{mathspec}
\usepackage{mathspec}
\usepackage[final]{listings}
\usepackage{graphicx}

\setallmainfonts{TeX Gyre Schola}
\setallmonofonts{Menlo}

%\usepackage{xunicode}
%\usepackage{xltxtra}
%\usepackage{graphicx}
%\usepackage{forloop}
%\usepackage{stmaryrd}
%\usepackage{mathtools}
%\usepackage{fancyvrb}

\usepackage{tikz}
% \usetikzlibrary{arrows.meta}
% \usetikzlibrary{backgrounds}
\usetikzlibrary{calc}
% \usetikzlibrary{datavisualization}
% \usetikzlibrary{datavisualization.formats.functions}
\usetikzlibrary{math}
% \usetikzlibrary{fit}
% \usetikzlibrary{positioning}
% \usetikzlibrary{scopes}
\usetikzlibrary{shapes.geometric}

\def\bigO#1{\ensuremath{\mathcal O(#1)}}
\def\BigO#1{\ensuremath{\mathcal O\left(#1\right)}}
\newcommand\fun[1]{\ensuremath{\lambda #1.\,}}
\renewcommand\T[1]{\ensuremath{T_{\mathit{#1}}}}

%%%
%%% BASIC SETTINGS
%%%

\frenchspacing
\unitlength=0.01\textwidth
\thicklines
\urlstyle{sf}
\graphicspath{{images/}}
\setlength{\parskip}{.25em}

\defaultfontfeatures{
    Mapping=tex-text,
    Scale=MatchLowercase,
}

%%%
%%% LISTINGS SETUP
%%%

\lstdefinelanguage{ASL}%
  {sensitive,%
   alsoletter=-+*/\#?,%
   alsodigit=-,%
   morekeywords={%
      and,%
      begin,%
      check-error,%
      check-expect,%
      check-within,%
      cond,%
      define,%
      define-struct,%
      else,%
      if,%
      lambda,%
      local,%
      or,%
   },%
   morekeywords=[2]{%
     cons,%
     \#true,%
     \#false,%
   },%
   morekeywords=[3]{%
     this-is-here-because-suck,%
     cons?,%
     empty?,%
     -,+,*,/,%
     first,%
     rest,%
   }%,
   morecomment=[l];,%
   morecomment=[s]{\#|}{|\#},%
   morestring=[b]"%
  }[keywords,comments,strings]%

\lstnewenvironment{ASL}[1][]
{%
  \begingroup
  \lstset{language=ASL,#1}%
}
{%
  \endgroup
}

\lstnewenvironment{CXX}[1][]
{%
  \begingroup
  \lstset{language=C++,#1}%
}
{%
  \endgroup
}

\lstnewenvironment{Java}[1][]
{%
  \begingroup
  \lstset{language=Java,#1}%
}
{%
  \endgroup
}

\lstnewenvironment{Python}[1][]
{%
  \begingroup
  \lstset{language=Python,#1}%
}
{%
  \endgroup
}


\lstnewenvironment{Ruby}[1][]
{%
  \begingroup
  \lstset{language=Ruby,#1}%
}
{%
  \endgroup
}

\newcommand{\textcode}{%
  \lstset{language=ASL,basicstyle=\ttfamily\small}%
  \lstinline
}

\lstset{%
  numberstyle=\scriptsize,
  showspaces=false,
  basicstyle=\ttfamily,
  keywordstyle=\ttfamily\color{red!70!black},
  keywordstyle=[2]\ttfamily\color{blue!70!black},
  keywordstyle=[3]\ttfamily\color{green!40!black},
  numberstyle=\ttfamily\color{blue!70!black},
  commentstyle=\itshape\color{violet!80!black},
  escapeinside={($}{$)},
}

\newcommand{\textasl}{%
  \lstset{language=ASL,basicstyle=\ttfamily\small}%
  \lstinline
}

\let\sup^

\catcode`\^\active\relax
\def^#1^{\textasl|#1|}

%%%
%%% BEAMER SETUP
%%%

\setbeamercolor{frametitle}{fg=black}
\setbeamercolor{normal text}{fg=black}

\usefonttheme{serif}

\setbeamertemplate{frametitle}
  {\begin{center}\medskip
   \insertframetitle\par
   \end{center}}

\setbeamertemplate{itemize item}{$\bullet$}

\setbeamertemplate{navigation symbols}{}
\setbeamertemplate{footline}[text line]{%
    \hfill\strut{%
        \small\color{black!75}%
        \texttt{\insertframenumber:\insertframeslidenumber}%
    }%
    \hfill%
}

\makeatletter
\newcount\frameslidetemp
\newcommand\insertframeslidenumber{%
  \frameslidetemp\c@page\relax
  \advance\frameslidetemp-\beamer@startpageofframe\relax
  \advance\frameslidetemp1\relax
  \the\frameslidetemp
}
\makeatother

%%%
%%% GENERALLY USEFUL LITTLE MACROS
%%%

% For using \alt within verbatim:
\newcommand\ALT[3]{\alt<#1>{#2}{#3}}

% Or: \ALT = \S\alt
\def\S#1#2{#2<#1>}

\newcommand<>\always[1]{#1}

\newcommand\widen[1][1.2]{%
    \setlength\baselineskip{#1\baselineskip}%
}

%%%
%%% COLORS
%%%

\newcommand\K[1]{\textcolor{orange!70!black}{#1}}
\newcommand\REF{\textcolor{green!50!black}{ref}}

%%%
%%% TIKZ
%%%

% \place[STYLE]{LOCATION}{CONTENT} places CONTENT at LOCATION, applying
% STYLE
\newcommand\place[3][]{%
  \tikz[orp] \node[#1] at (#2) {#3};%
}

% \anchor{NAME} creates an addressable TikZ coordinate at the current
% location.
\newcommand\anchor[1]{%
  \tikz[orp] \coordinate (#1);%
}

% \nil{NODE} draws a slash across the node; if NODE is a square then we
% get the standard graphic notation for null/nil.
\newcommand\nil[1][1em]{%
  \begin{tikzpicture}[nil]
    \node[pointer tail, line width=0, opacity=0] {};
    \draw[overlay]
      node[pointer tail] (dummy) {}
      (dummy.west |- dummy.south) -- (dummy.east |- dummy.north);
  \end{tikzpicture}%
}

\makeatletter
  \newcommand\cons[1][]{\cons@kont{#1}}
  \def\cons@kont#1#2(#3)#4#5{\struct@kont{#1}{}(#3){#4&#5}}

  \newcommand\struct[1][]{\struct@kont{#1}}
  \def\struct@kont#1#2(#3)#4{
      \node[draw, inner sep=0, #1] (#3) {%
        \begin{tabular}{|*{100}{c|}}
          \hline
          #4\\
          \hline
        \end{tabular}
      }
  }
\makeatother

% \genpointer[STYLE]{BEND}{TARGET}
\newcommand\genpointer[2][genpointer]{%
  \tikz[remember picture, #1]
    \draw
      node[pointer tail] {}
      edge[->, overlay] (#2) ;%
}

% \hpointer[BEND]{TARGET} for a highlighted pointer
\newcommand\hpointer[2][]{%
  \genpointer[highlit pointer,#1]{#2}%
}

% \ppointer[BEND]{TARGET} for a non-highlighted pointer
\newcommand\ppointer[2][]{%
  \genpointer[pointer,#1]{#2}%
}

% \pointer[BEND]{TARGET} is highlighted if used in a highlighted context
\newcommand\pointer[2][]{%
  \ifhighlit\let\pointertemp\hpointer\else\let\pointertemp\ppointer\fi
  \pointertemp[#1]{#2}%
}

\newcommand\genpointerbase[2][genpointer]{%
  \tikz[remember picture, #1]
    \draw node[pointer tail] (#2) {};%
}

\newcommand\ppointerbase[2][genpointer]{%
  \genpointerbase[pointer,#1]{#2}%
}

\newcommand\hpointerbase[2][genpointer]{%
  \genpointerbase[highlit pointer,#1]{#2}%
}

\newcommand\pointerbase[2][]{%
  \ifhighlit\let\pointertemp\hpointerbase\else\let\pointertemp\ppointerbase\fi
  \pointertemp[#1]{#2}%
}

\newcommand\genpointerpoint[3][genpointer]{
  \draw (#2) edge[->, remember picture, overlay, #1] (#3) ;
}

\newcommand\ppointerpoint[2][genpointer]{%
  \genpointerpoint[pointer,#1]{#2}%
}

\newcommand\hpointerpoint[2][genpointer]{%
  \genpointerpoint[highlit pointer,#1]{#2}%
}

\newcommand\pointerpoint[2][]{%
  \ifhighlit\let\pointertemp\hpointerpoint\else\let\pointertemp\ppointerpoint\fi
  \pointertemp[#1]{#2}%
}

\tikzset{
  onslide/.code args={<#1>#2}{%
    \only<#1>{\pgfkeysalso{#2}}
  },
  orp/.style={
    overlay,
    remember picture,
  },
  rem/.style={
    remember picture,
  },
  every node/.style={
    very thick,
    inner sep=3pt,
  },
  treenode/.style={
    circle,
    draw=black,
    inner sep=2pt,
  },
  subtree/.style={
    treenode,
    isosceles triangle,
    isosceles triangle apex angle=30,
    text width=1em,
    align=center,
    shape border rotate=90,
    anchor=apex,
    inner xsep=0pt,
    inner ysep=4pt,
    fill=orange!30!white,
    draw=orange!50!black,
  },
  balance/.style={
    draw=none,
    color=green!70!black,
    anchor=north,
  },
  bad balance/.style={
    balance,
    color=red,
    },
  every path/.style={very thick},
  ref/.style={
    draw=red,
    densely dotted,
    font=\ttfamily,
  },
  cons/.style={
    draw=black,
    font=\ttfamily,
  },
  genpointer/.style={
    solid,
    bend left=30,
    >=Latex,
  },
  pointer/.style={
    genpointer,
    color=blue!50!black,
    opacity=.5,
  },
  highlit pointer/.style={
    genpointer,
    color=red!70!black,
  },
  nil/.style={
    color=black!50,
  },
  pointer tail/.style={
    draw,
    circle,
    text width=0,
    text height=1ex,
    inner sep=0,
  },
  highlight base/.style={
    outer color=#1!50,
    inner color=#1!30,
  },
  highlight base/.default=yellow,
  highlight/.style={
    highlight base,
    color=yellow!50!white!90!black,
    draw opacity=0.5,
    thick,
    draw,
    rounded corners=2pt,
  },
  bintree/.style={
    level 1/.style={sibling distance=#1},
    level 2/.style={sibling distance=#1/2},
    level 3/.style={sibling distance=#1/4},
    level 4/.style={sibling distance=#1/8},
    level 5/.style={sibling distance=#1/16},
  },
  bintree/.default={8em},
}

\newif\ifhighlit
\highlitfalse

% \highlight<WHEN>[NAME]{CONTENT}
\newcommand<>\highlight[2][\relax]{%
  \ifx#1\relax
    \begin{tikzpicture}
  \else
    \begin{tikzpicture}[remember picture]
  \fi
    \node[
      inner sep=0pt,
      outer sep=0pt,
      text depth=0pt,
      align=left,
    ] (highlight temp) {%
        \only#3{\highlittrue}%
        #2%
        \only#3{\highlitfalse}%
    };
    \begin{scope}[on background layer]
      \coordinate (highlight temp 1) at
        ($(highlight temp.north east) + (2pt,2pt)$);
      \coordinate (highlight temp 2) at
        ($(highlight temp.base west) + (-2pt,-3pt)$);
      \node#3 [
        overlay,
        highlight,
        inner sep=0pt,
        fit=(highlight temp 1) (highlight temp 2),
      ] {};
    \end{scope}
    \ifx#1\relax
      \relax
    \else
      \node (#1) [
        overlay,
        inner sep=0pt,
        fit=(highlight temp 1) (highlight temp 2),
      ] {};
    \fi
  \end{tikzpicture}%
}

\newcommand\HI[2]{\highlight<#1>{#2}}

% \huncover[NAME]{STEP}{CONTENT} uncovers from STEP onward, highlighting
% at STEP but not thereafter.
\newcommand\huncover[3][\relax]{%
  \uncover<#2->{\highlight<#2>[#1]{#3}}%
}

% \huncover[NAME]{STEP}{CONTENT} onlys from STEP onward, highlighting
% at STEP but not thereafter.
\newcommand\honly[3][\relax]{%
  \only<#2->{\highlight<#2>[#1]{#3}}%
}

\begin{document}

%%%
%%% TITLE PAGE
%%%

\newcommand{\takeaways}{
  \begin{frame}{Take-aways}
    \begin{itemize}
      \item How are graphs commonly represented?
      \item How can we find shortest paths?
    \end{itemize}
  \end{frame}
}

\begin{frame}[fragile]\relax
  \thispagestyle{empty}

  %% Need to initialize Tikz opacity inside a frame, not outside:
  \begin{tikzpicture}[opacity=0,overlay]\end{tikzpicture}

  \begin{center}
    {\huge Graph representations and algorithms}

  \bigskip
  \large
  EECS 214

  \medskip
  November 23, 2015
  \end{center}
\end{frame}

\takeaways

\begin{frame}{Definitions}
  A \emph{graph} is a pair $(V, E)$, where $V$ is the set of
  \emph{vertices} and symmetric relation $E \subseteq V\sup2$ is
  the set of \emph{edges}.
  \par\pause\bigskip
  A \emph{directed graph} is a pair $(V, E)$ where $V$ is the set of
  \emph{vertices} and relation $E \subseteq V\sup2$ is the set of edges.
\end{frame}

\begin{frame}{Weighted graph definitions}
  Let $W$ be a set of \emph{weights}. (Often $W = \mathbb R$.) Then:
  %
  \par\bigskip
  A \emph{weighted directed graph} is a pair $(V, E)$, where $V$ is the
  set of \emph{vertices} and $E : V\sup2 \rightharpoonup W$ is a
  partial map from edges to their weights.
  %
  \par\medskip\pause
  Alternatively, we can make $E$ total: let $\infty \in W$ and $E :
  V\sup2 \to W$. Then $E(v, u) = \infty$ means there is no edge from $v$
  to $u$.
  \par\bigskip\pause
  A \emph{weighted (undirected) graph} is a weighted directed graph where
  $E(v, u) = E(u, v)$ for all $v, u \in V$.
\end{frame}

\begin{frame}[fragile]{Two major graph representations}
  \begin{center}
  \begin{tikzpicture}
    [
      every node/.style={
        draw,
        circle,
      },
    ]
    \node (A) at (0:1.4) {A};
    \node (B) at (60:1.4) {B};
    \node (C) at (120:1.4) {C};
    \node (D) at (180:1.4) {D};
    \node (E) at (240:1.4) {E};
    \node (F) at (300:1.4) {F};
    \draw[->] (A) edge (B) edge (C)
              (B) edge (F)
              (C) edge (D) edge (F)
              (D) edge (E) edge (A)
              (E) edge (F) edge (C)
              ;
  \end{tikzpicture}
  \end{center}
  \uncover<3->{
  \begin{minipage}{2in}
    \centering
    \textbf{Adjacency matrix}\\[4pt]
  \begin{tabular}{c||c|c|c|c|c|c}
    & A & B & C & D & E & F \\
    \hline
    \hline
    A & & \textbullet & \textbullet & & & \\
    \hline
    B & & & & & & \textbullet \\
    \hline
    C & & & & \textbullet & & \textbullet \\
    \hline
    D &\textbullet & & & &\textbullet &  \\
    \hline
    E & & &\textbullet & & &\textbullet  \\
    \hline
    F & & & & & &  \\[4pt]
   \end{tabular}
  \end{minipage}~%
  }%
  \uncover<2->{%
  \begin{minipage}{2in}
    \centering
    \textbf{Adjacency list}\\
  \begin{tabular}{c|l}
    V & Successors \\
    \hline
    A & B, C \\
    B & F    \\
    C & D, F \\
    D & A, E \\
    E & F, C \\
    F &      \\
   \end{tabular}
  \end{minipage}
}
\end{frame}

\begin{frame}[fragile]{Two major weighted graph representations}
  \begin{center}
  \begin{tikzpicture}
    [
      vertex/.style={
        draw,
        circle,
      },
    ]
    \node[vertex] (A) at (0:1.4) {A};
    \node[vertex] (B) at (60:1.4) {B};
    \node[vertex] (C) at (120:1.4) {C};
    \node[vertex] (D) at (180:1.4) {D};
    \node[vertex] (E) at (240:1.4) {E};
    \node[vertex] (F) at (300:1.4) {F};
    \draw[->] (A) edge node[right]{2} (B) edge node[above]{3} (C)
              (B) edge (F)
              (C) edge node[left]{1} (D) edge (F)
              (D) edge node[left]{1} (E) edge (A)
              (E) edge node[left]{1} (F) edge (C)
              ;
  \end{tikzpicture}
  \end{center}
  \uncover<3->{
  \begin{minipage}{2in}
    \centering
    \textbf{Adjacency matrix}\\[4pt]
  \begin{tabular}{c||c|c|c|c|c|c}
    & A & B & C & D & E & F \\
    \hline
    \hline
    A & & \textbullet & \textbullet & & & \\
    \hline
    B & & & & & & \textbullet \\
    \hline
    C & & & & \textbullet & & \textbullet \\
    \hline
    D &\textbullet & & & &\textbullet &  \\
    \hline
    E & & &\textbullet & & &\textbullet  \\
    \hline
    F & & & & & &  \\[4pt]
   \end{tabular}
  \end{minipage}~%
  }%
  \uncover<2->{%
  \begin{minipage}{2in}
    \centering
    \textbf{Adjacency list}\\
  \begin{tabular}{c|l}
    V & Successors \\
    \hline
    A & B, C \\
    B & F    \\
    C & D, F \\
    D & A, E \\
    E & F, C \\
    F &      \\
   \end{tabular}
  \end{minipage}
}
\end{frame}

\begin{frame}
  \begin{center}
    \emph{Dijkstra's algorithm}
  \end{center}
\end{frame}
\takeaways

\end{document}
% \begin{frame}[fragile]{Graph search (basic algorithm)}
% \begin{tabbing}
% \textbf{def} GraphSearch(\emph{start}): \\
% \quad\=\+
% \emph{visited} \=$\gets$ $\varnothing$ \\
% \emph{todo}    \>$\gets$ $\{$ \emph{start} $\}$ \\
% \\
% \textbf{while} \emph{todo} $\ne$ $\varnothing$: \\
% \quad\=\+
% $v \gets$ remove an element from \emph{todo} \\
% \textbf{if} \emph{v} $\not\in$ \emph{visited}: \\
% \quad\=\+
% Visit$(v)$ \\
% \emph{visited} \=$\gets$ $\{v\}$ $\cup$ \emph{visited} \\
% \emph{todo}    \>$\gets$ Successors$(v)$ $\cup$ \emph{todo} \\
% \end{tabbing}
% \end{frame}

% \begin{frame}[fragile]{Breadth-first search}
% If we make \emph{todo} a queue (FIFO), we get BFS:
% \begin{tabbing}
% \textbf{def} BFS(\emph{start}): \\
% \quad\=\+
% \emph{visited} \=$\gets$ $\varnothing$ \\
% \emph{todo}    \>$\gets$ empty queue \\
% enqueue \emph{start} in \emph{todo} \\
% \\
% \textbf{while} \emph{todo} $\ne$ $\varnothing$: \\
% \quad\=\+
% $v \gets$ dequeue an element from \emph{todo} \\
% \textbf{if} \emph{v} $\not\in$ \emph{visited}: \\
% \quad\=\+
% Visit$(v)$ \\
% \emph{visited} \=$\gets$ $\{v\}$ $\cup$ \emph{visited} \\
% enqueue \only<2>{(}Successors$(v)$\only<2>{ $\setminus$ \emph{visited})} in \emph{todo} \\
% \end{tabbing}
% \uncover<2>{\emph{(an optimization)}}
% \end{frame}

% \begin{frame}[fragile]{Depth-first search}
% If we make \emph{todo} a stack (LIFO), we get DFS:
% \begin{tabbing}
% \textbf{def} BFS(\emph{start}): \\
% \quad\=\+
% \emph{visited} \=$\gets$ $\varnothing$ \\
% \emph{todo}    \>$\gets$ empty stack \\
% push \emph{start} onto \emph{todo} \\
% \\
% \textbf{while} \emph{todo} $\ne$ $\varnothing$: \\
% \quad\=\+
% $v \gets$ pop an element from \emph{todo} \\
% \textbf{if} \emph{v} $\not\in$ \emph{visited}: \\
% \quad\=\+
% Visit$(v)$ \\
% \emph{visited} \=$\gets$ $\{v\}$ $\cup$ \emph{visited} \\
% push Successors$(v)$ onto \emph{todo} \\
% \end{tabbing}
% \end{frame}

